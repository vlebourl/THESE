\chapter{Resilience of population's size structure: experimental modification of
size structure and its impact on the size structure dynamics}

\section{Introduction}

\section{Material and methods}
\subsection{Modification of size structure}

\lettrine[lines=3]{T}{o study the impact} of a sudden change in size
distribution and density on the dynamics of the populations and their size structure, we monitore 8 additional
populations of each HA and TO clone. The populations are kept in the same
conditions as previously described for twelve months. After twelve months, we
consider that the transiant dynamics of the size structure is over. For each
clone, we keep two populations with the same size structure as control
populations that we moved to new rearing boxes. The remaining six populations
are split in two according to the size of the individuals.
Our populations of the clone HA exhibit size structures with three distinct
modes, juveniles, small adults and large adults. We seperate our six populations
into three groups of two replicates. (1) In the first group, we remove the
juveniles that we use to start new populations, moving the small adults and the
big adults together in new rearing boxes. (2) In the second group, we keep
together the juveniles and the large adults, removing the small adults to start
new populations. (3) Finally, in the last group, we separete the large adults
from the rest of the populations, leaving together the juveniles and the small
adults.

Our populations of the clone TO have only bimodal size distributions. To compare
with our populations of clone HA, we seperated the adult class in two groups
constituted respectively with two third of the adults to mimic the impact of a
removal of the small adults, and one third of the adult to mimic the large adult
group in clone HA. We then split the population following the same protocole as
for clone HA. (1) We isolate the juveniles to start new populations, keeping all
the adults together in new rearing boxes. (2) We keep together the juveniles and
one third of the adults, moving the remaining two third of the adults in new
populations. And (3), we remove one third of the adults to start new
populations, keeping together the juveniles and the remaining third of the
adults. These manipulations of the size structure are summurized in Figure 1.
Once the populations are splitted, we keep them in the same rearing conditions
for 15 months with weekly numbering and mesurements of the individuals' body
length. We analyze the impact of the perturbation of the size-structure and the
density on the dynamics of the size structure using the same graphical
representation as for the long term population surveys described previously.

\subsection{Behavioral observation close to the resource}

In order to better understand how the resources are shared between the
individuals of a population, we looked at grazing behaviour when the resource is
available for a population.
One observation consiste of the following. First, we add resource in the rearing
box, as centered in the box as possible. We moisten the yeast and agar pellet we
add, so that it sticks uniformelly to the plaster and individuals can only
access the resource from the top or the side of the pellet. That way all the
individuals accessing the resource are always visble. We then observe
continuously the region with the pellet using a numerical USB microscope
(Dino-Lite). We take snapshots of the pellet and its surrounding every 15
minutes during the whole time needed for its complete consumption. Finally, we
select a set of 10 consecutive snapshots on which the pellets is evenly
recovered by individuals, and the diameter of the pellet is constant over the 10
pictures.
We use the software ImageJ to number and measure every single individual
accessing the resource visible on a snapshot. To do that properly, we define a
standard circle region around the pellet that we split in halves. We consider
individuals as accessing the resource when, on the left side, they are strictly
inside the circle, and on the right hand side when they are inside or touch the
circle. We repeat the individual measurement for each 10 pictures of an
observation, constituting 10 replicates of the measurment of the individuals
accessing the resource that we consider independant.
We repeated this measure on five times for the clone HA and five times for the
clone TO on populations from the size structure modification experiment,
allowing to make the observation for different size structures for both clones.
Each observation then allows to determine the size-structure of the subset of
the population accessing the resource and compare it to the distribution of body
length in the original population.