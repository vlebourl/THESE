\chapter[Interférence vs. exploitation et dynamique des populations
structurées][Interférence et populations structurées]{Interférence vs.
exploitation et dynamique des populations structurées}
\label{chap:amnat}

\vspace{2cm}
\begin{Spacing}{1}
\texttt{
Le Bourlot, Vincent, Thomas Tully and David Claessen, "Interference versus
Exploitative Competition in the regulation of Size-Structured Populations"\\
under review at The American Naturalist
}
\end{Spacing}

\section*{Résumé}
%\addcontentsline{toc}{section}{Résumé}


\lettrine[lines=3]{L}{a compétition}  est un des facteurs principaux dans la
régulation de la dynamique des populations et des communautés. Son effet peut
être soit direct entre plusieurs individus via la compétition par
interférence, ou par l'intermédiaire de la ressource dans la compétition par
exploitation. L'impact de la compétition par exploitation sur la dynamique des
populations a déjà été largement étudié, tant d'un point de vue empirique
que théorique, mais les effets de la compétition par interférence restent
quant à eux mal compris. Nous étudions ici les effets de différents niveaux
de compétition intra-spécifique par interférence sur la dynamique d'une
population structurée en taille. Nous basons notre étude sur un modèle
ressource -- consommateur physiologiquement structuré prenant en compte des
interactions directes entre les individus, autorisant ainsi un gradient de
compétition depuis une compétition purement par exploitation à une
compétition totalement dominée par l'interférence. Nous paramétrons notre
modèle en utilisant des données issues de suivis expérimentaux de populations
de collemboles \textit{Folsomia candida}. Notre modèle prédit une variété de
dynamiques possibles suivant le niveau de compétition par interférence
imposé. A un faible niveau d'interférence, notre modèle se comporte de
manière similaire au modèle classique de Kooijman et Metz. Un niveau plus fort
d'interférence agit comme une force stabilisatrice sur les cycles de
générations causés par les juvéniles. A niveau intermédiaire, des géants
émergent dans la populations et commencent à la dominer. Enfin, à un niveau
très élevé d'interférence, un nouveau type de cycle apparaît que l'on
appelle cycles causés par l'interférence. Nos résultats théorique permettent
d'apporter un nouvel éclairage dans l'interprétation des dynamiques de la
structure en taille des populations de collemboles élevées au laboratoire.

\section{Introduction et méthodologie}

Dans ce chapitre, nous nous intéressons aux compétitions par exploitation et
par interférence, et à leur conséquences sur la dynamique prédite par un modèle
de populations structurées.


