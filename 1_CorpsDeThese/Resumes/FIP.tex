\chapter{Interactions entre compétition intraspécifique et température dans la
régulation des populations structurées}
\chaptermark{Compétition intraspécifique et température}
\label{chap:fip}

\vspace{2cm}
\begin{Spacing}{1}
\texttt{
Mallard, François, Vincent Le Bourlot, Christie Le Coeur, Monique Avnaim, David
Claessen and Thomas Tully, "From individuals to populations: intraspecific
competition breaks the temperature-size rule"\\
soumis à Journal of Animal Ecology}
\end{Spacing}
\vspace{2cm}


\lettrine[lines=3]{A}{u cours} des trois chapitres précédents, nous avons étudié
le rôle des mécanismes de compétition intraspécifiques sur la dynamique de
populations structurées de collemboles \textit{Folsomia candida} d'un point de vu théorique
et empirique. Ces études nous ont permis en particulier de mieux appréhender
l'impact de différents niveaux de compétition par interférence dans la
dynamique temporelle de la structure des populations. Cependant, ces études ont
été réalisées dans une seule condition environnementale. Hors, comme nous
l'avons déjà expliqué, l'environnement, et en particulier la température, peut
avoir un impact à la fois sur les individus et les populations. 

En effet, des températures différentes affectent directement les taux
physiologiques (métabolisme, consommation, respiration,\ldots), ce qui a des
conséquences démographiques via des changements dans les cycles de vie
\autocites{gillooly2002a,le-galliard2012a}. des changements de température
provoque aussi de la plasticité phénotypique chez les individus en modifiant par
exemple l'allocation des ressources à la croissance ou à la reproduction
\autocites{liefting2010temperature,gutteling2007mapping}. Ceci résulte notamment
dans la règle dite ``taille-température'' qui prédit une taille plus grande des
individus dans des environnements plus froids
\autocites{atkinson1994a,atkinson1996a,angilletta2009a}. Enfin, la température
peut également avoir un effet sur les comportements individuels comme
l'activité,la dispersion ou le choix de l'habitat
\autocites{atacho2013a,bonte2008thermal,vanbeest2012temperature}.

L'approche classique de l'étude de l'influence de la température sur les
phénotypes consiste à mesurer des normes de réaction \autocites{woltereck1909a}.
Ces mesures sont généralement réalisées sur des individus isolés ou sur de
petites cohortes élevées au laboratoire dans différentes conditions de
température. Ces analyses apportent beaucoup d'informations sur les effets au
niveau de l'individu mais laissent de côté les conséquences de ces effets aux
niveaux des populations et des communautés. 

Dans ce chapitre, nous nous demandons a quel point les normes de réactions
mesurées au niveau individuel permettent des inférences sur la dynamique des
populations. Nous cherchons à comprendre comment les effets directs de la
température sur les traits d'histoire de vie individuels sont modulés par les
interactions entre individus et les rétroactions démographiques.
Les effets d'une augmentation de la température sur la compétition
inter-spécifique ont déjà été démontrés, provoquant notamment une augmentation
de la compétition par exploitation. Mais le cas de la compétition
intra-spécifique reste peu clair, et les effets sur la compétition par
interférence peuvent être différents de l'exploitation. L'interaction entre
les deux mécanismes, couplés aux effets directes de la température sur les individus,
rendent l'impact de la température sur les dynamiques des populations
structurées difficile à prévoir. Pour y parvenir, nous avons mesuré les normes de réaction
individuelles des taux de croissance et des tailles à maturité et tailles
asymptotiques dans quatre conditions de température, $11$, $16$, $21$ et
$26\degres$C. Nous avons également effectué des mesures de taux de croissance de
cohortes et de taille des cohortes adultes dans des populations élevées aux même
températures. Nous avons ainsi pu comparer les réponses à la température dans
deux conditions démographiques contrastées et en déduire l'importance de la
compétition intra-spécifique dans la réponse des populations à la température.

\section{Éléments de méthodologie}

\subsection{Conditions d'élevages et mesures}

Comme dans les expériences précédentes, les conditions d'élevage et les méthodes
de mesure de la taille individuelle et de la structure des populations suivent
les descriptions données dans le Chapitre \ref{chap:method}.

\subsubsection{Individus isolés}

Les nouveaux nés sont isolés immédiatement après la naissance et sont nourris
\textit{ad libitum} pendant toute leur vie. Les individus ont été placé
aléatoirement à une des quatre températures de notre intervalle ($11$, $16$, $21$ et
$26\degres$C). La taille corporelle est mesurée pour chaque individu trois fois
par semaine pendant dix semaines, puis une fois par semaine. Afin de déterminer
la maturation des individus, les boites d'élevage sont régulièrement inspecter
pour rechercher des pontes. La croissance des individus est ensuite modélisée
par une fonction logistique ajustée sur les mesures de taille. Le taux de
croissance maximal moyen et la taille asymptotique sont estimés par des modèles
de moindre-carrés non linéaire ajustés séparément pour chacune des trajectoires
de croissance \autocites{pinheiro2000a}. Les effets fixes du clone et de la
température, prise comme variable catégorielle, sur le taux de croissance
maximal et la taille asymptotique ont été testés à l'aide de modèles linéaires
et de test de Fisher.

\subsubsection{Mesures de taille et de croissance dans les populations}
