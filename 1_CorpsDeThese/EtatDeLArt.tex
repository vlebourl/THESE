\chapter{État de l'art}

\section{Les conséquences écologiques de la structuration des populations}

\lettrine[lines=3]{P}{our comprendre} le fonctionnement des écosystèmes et les
réponses des espèces à leur environnement, il est d'important de comprendre leur démographie et la
dynamique de leur populations. De nombreuses études empiriques ont montré que
ces populations étaient structurées de façon non triviales. C'est un 
résultat très général qui a été vérifié à de nombreuses reprises, que ce soit en
laboratoire, comme chez la drosophile ou chez des acariens par exemple, ou dans
des populations naturelles telles que les populations de moutons de Soay ou de cerfs
élaphe. 
Cela implique que la description d'une population comme un tout ou comme
un assemblage de classe crées artificiellement représente généralement mal la
réalité et n'intègre pas suffisamment de complexité pour décrire fidèlement les
mécanismes qui régulent sa dynamique.

\subsection{Différents niveaux de structuration}

Dans une population, la structure émerge de l'hétérogénéité entre les
individus d'une même espèce. Plusieurs formes de structures ont été
classiquement prises en compte en écologie.

\subsubsection{Structuration spatiale}

Une première forme de structuration évidente est la structuration spatiale. Ceci
décrit comment les individus d'une population s'organisent dans l'espace, et ce
faisant, modifient leurs interactions entre eux et avec leur environnement. 

La structuration spatiale des populations répond souvent à l'hétérogénéité de
leur habitat. Ces hétérogénéités ont des conséquences directes sur la dynamiques
des populations, par exemple en modifiant les schémas de dispersion des
individus \autocite{hiebeler2000a}, leur fitness \autocite{zajkac2008a},
l'accès aux ressources \autocite{burger2008a}, la sensibilité aux parasite ou
pathogènes \autocite{su2009a}, \textit{etc}.

L'étude de la dynamique des populations structurées spatialement constitue un
champs de recherche extrêmement large et varié auquel notre étude ne se rattache
pas directement. 

\subsubsection{Structuration génétique}

Conséquence de la structuration spatiale, les population sont souvent également
structurées génétiquement. Les individus les plus proches les uns des autres,
notamment dans des méta-populations, sont également plus proches génétiquement
que des individus spatialement éloignés. L'analyse génétique d'une population
permet alors d'obtenir des informations sur ses origines et sa structuration
spatiale \autocite{repaci2006a,booth2009a,jorde2007a}.

\subsubsection{Structuration en stades}
\subsubsection{Structuration en âges}
\subsubsection{Structuration physiologique}

\subsection{Importance de la taille corporelle}

\subsubsection{Sur les traits d'histoire de vie}

\subsubsection{Sur la structure et la dynamique des populations}

\subsubsection{Sur la structure et la dynamique des communautés}





% Une des causes principales de cette hétérogénéité
% vient du cycle de vie de l'espèce et des différentes étapes qu'un individu
% traverse au cours de ce cycle. 

\section{Mécanismes de densité dépendance}

\section{Le rôle de la température}

\section{Problématiques}	
