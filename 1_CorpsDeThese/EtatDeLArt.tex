\chapter{État de l'art}

\section{Les conséquences écologiques de la
structuration des populations}
\sectionmark{Structuration des populations}

\lettrine[lines=3]{P}{our comprendre} le fonctionnement des écosystèmes et les
réponses des espèces à leur environnement, il est d'important de comprendre leur démographie et la
dynamique de leur populations. De nombreuses études empiriques ont montré que
ces populations étaient structurées de façon non triviales. C'est un 
résultat très général qui a été vérifié à de nombreuses reprises, que ce soit en
laboratoire, comme chez la drosophile ou chez des acariens par exemple, ou dans
des populations naturelles telles que les populations de moutons de Soay
\autocite{coulson2001a} ou de cerfs élaphe. 
Cela implique que la description d'une population comme un tout ou comme
un assemblage de classe crées artificiellement représente généralement mal la
réalité et n'intègre pas suffisamment de complexité pour décrire fidèlement les
mécanismes qui régulent sa dynamique.

\subsection{Différents niveaux de structuration}

Dans une population, la structure émerge de l'hétérogénéité entre les
individus d'une même espèce. Plusieurs formes de structures ont été
classiquement prises en compte en écologie.

\subsubsection{Structuration spatiale}

Une première forme de structuration évidente est la structuration spatiale. Ceci
décrit comment les individus d'une population s'organisent dans l'espace, et ce
faisant, modifient leurs interactions entre eux et avec leur environnement. 

La structuration spatiale des populations répond souvent à l'hétérogénéité de
leur habitat. Ces hétérogénéités ont des conséquences directes sur la dynamiques
des populations, par exemple en modifiant les schémas de dispersion des
individus \autocite{hiebeler2000a}, leur fitness \autocite{zajkac2008a},
l'accès aux ressources \autocite{burger2008a}, la sensibilité aux parasite ou
pathogènes \autocite{su2009a}, \textit{etc}.

L'étude de la dynamique des populations structurées spatialement constitue un
champs de recherche extrêmement large et varié auquel notre étude ne se rattache
pas directement. 

\subsubsection{Structuration génétique}

Conséquence de la structuration spatiale, les population sont souvent également
structurées génétiquement. Les individus les plus proches les uns des autres,
notamment dans des méta-populations, sont également plus proches génétiquement
que des individus spatialement éloignés. L'analyse génétique d'une population
permet alors d'obtenir des informations sur ses origines et sa structuration
spatiale \autocite{repaci2006a,booth2009a,jorde2007a}.

\subsubsection{Structuration en stades}

Une des causes principales de l'hétérogénéité à l'origine de la
structuration des populations vient du cycle de vie des individus. Lorsque le
cycle de vie d'une espèce est tel que les traits d'histoire de vie comme la
croissance, la reproduction ou la mortalité varient beaucoup entre des étapes
différents mais sont très similaires au sein d'une même étape, on peut alors
séparer la population en plusieurs stades définis par les différentes étapes du
cycle de vie.

Cette forme de structuration, intégrée dans des modèles de dynamique de
population depuis une trentaine d'année \autocite{gurney1983a,nisbet1983a},
permettent une description plus rigoureuse des relations entre les traits
d'histoire de vie individuelle et la dynamique de la population. Cette forme de
structuration et les modèles qui en découlent ont principalement été appliqués à
des populations d'invertébrés et d'insectes dont le cycle de vie contient un ou
plusieurs événements de métamorphose
\autocite{gurney1980a,gurney1983a,nisbet1983a,nisbet1989a,mccauley1996a}.

\subsubsection{Structuration en âges}

La structuration en âge d'une population est également couramment utilisée en
dynamique des populations lorsque l'âge de l'individu devient l'unité pertinente
pour suivre les variations des traits d'histoire de vie. L'âge des individus est
maintenant très couramment incorporé lors des études de dynamique de populations
naturelles ou théoriques
\autocite[par
ex. ][]{coulson2008a,marteinsdottir2002a,worden2010a,robinson2013a}. 
Cependant, considérer une structuration par l'âge uniquement oblige à fixer pour
tous les individus une même progression dans les trajectoires de vie. Or, il
peut exister des différences d'histoire de vie entre deux individus du même âge
dans une même population. 

\subsubsection{Structuration physiologique}

Afin de palier à ce défaut, les écologues ont considéré des caractères
physiologiques comme éléments structurants des populations. De cette façon, les
traits d'histoire de vie des individus n'ont pas besoin d'être divisibles en des
classes bien distinctes, mais on tient néanmoins compte de l'impact de l'état
physiologique de l'individu sur ses traits d'histoire de vie que sont par
exemple la reproduction, la croissance, la mortalité ou la vitesse d'ingestion
de l'énergie. 

Un cadre théorique complet a été développé pour permettre d'étudier la dynamique
des populations structurées physiologiquement. Les modèles de
population physiologiquement structurés (modèles PSP pour "Physiologically
Structured Population") constituent une part importante de ce cadre théorique
\autocite{metz1986a,de-roos1992a,de-roos1997a}, et permettent de tenir compte
des histoires de vie dans les quelles les traits physiologiques et les
interactions écologiques varient de façon continue. 

Une sous partie des modèles PSP s'intéressent particulièrement au rôle de la
taille corporelle dans les interactions écologiques, l'histoire de vie, et les
répercutions sur la dynamique des populations. 

\subsection{Importance de la taille corporelle}

La taille corporelle constitue un facteur clé dans la compréhension des rapports
entre état individuel et traits d'histoire de vie, et leur conséquences sur la
dynamique des populations et des communautés. 

\subsubsection{Sur les traits d'histoire de vie}

L'influence de la taille corporelle sur les performances écologiques, mesurées
notamment par les taux vitaux (croissance, reproduction ou mortalité) ou les
interactions trophiques, on fait l'objet d'un grand nombre d'études théoriques
et expérimentales
\autocite[][,\ldots]{peters1986a,calder1996a,de-roos2001a,claessen2004a}. Par
exemple, des individus plus larges vont généralement être plus efficaces dans
leur recherche de nourriture, pouvoir se nourrir de proies plus grandes et
courir un risque réduit de prédation comparé à des individus plus petits
\autocite{paradis1996a}.
Dans un autre registre, les capacités de recherche de nourriture de la daphnée
en fonction de sa taille ont été mesurées en détail dans un grand nombre de
conditions différentes, et reliées à l'allocation de l'énergie assimilée à la
croissance, à la reproduction ou au métabolisme \autocite[par ex.
][]{lampert1978a,gurney1990a,mccauley1990a,kooijman2000a}. De nombreuses études
se sont également intéressées au comportement de recherche de nourriture et à la
gestion de l'énergie chez des populations de poissons \autocite[par ex.
][]{elliott1975a,mittelbach1981a,fuiman1994a,hjelm2001a}. Ces différentes études
expérimentales ont conduit au développement d'un modèle générique, au niveau
individuel, des traits d'histoire de vie tels que la capacité à rechercher de la
nourriture, la croissance et le développement, qui ne repose pas sur une espèce
en particulier \autocite{kooijman2000a,nisbet2000a,west2001a}. Dans ses travaux,
Kooijman \textcite{kooijman2000a} un cadre théorique à la fois concis et
élaboré, le budget énergétique dynamique ("dynamic energy budget"), qui décrit
la consommation  de l'énergie et des nutriments à l'échelle de l'individu en
relation avec sa taille corporelle, et leurs utilisation pour les différents
traits d'histoire de vie.

\subsubsection{Sur la structure et la dynamique des populations}

\subsubsection{Sur la structure et la dynamique des communautés}






\section{Mécanismes de densité dépendance}

\section{Le rôle de la température}

\section{Problématiques}	
