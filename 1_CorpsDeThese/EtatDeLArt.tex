\chapter{État de l'art}

\section{Populations structurées}

\lettrine[lines=3]{P}{our comprendre} le fonctionnement des écosystèmes et les
réponses des espèces à leur environnement, il est d'important de comprendre leur démographie et la
dynamique de leur populations. De nombreuses études empiriques ont montré que
ces populations étaient structurées de façon non triviales. Cela semble un 
résultat très général qui a été vérifié à de nombreuses reprises, que ce soit en
laboratoire, chez la drosophile ou chez des acariens par exemple, ou dans des
populations naturelles telles que les populations de moutons de Soay ou de cerfs
élaphe. Cela implique que la description d'une population comme un tout ou comme
un assemblage de classe crées artificiellement représente généralement mal la
réalité et n'intègre pas suffisamment de complexité pour décrire fidèlement les
mécanismes qui régulent sa dynamique. 

Dans une population, la structure émerge de l'hétérogénéité entre les
individus. Une des causes principales de cette hétérogénéité vient du cycle de
vie 

\section{Mécanismes de densité dépendance}

\section{Le rôle de la température}

\section{Problématiques}	
