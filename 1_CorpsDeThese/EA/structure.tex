\section{Les conséquences écologiques de la
structuration des populations}
\sectionmark{Structuration des populations}

\lettrine[lines=3]{P}{our comprendre} le fonctionnement des écosystèmes et les
réponses des espèces à leur environnement, il est d'important de comprendre leur démographie et la
dynamique de leurs populations. De nombreuses études empiriques ont montré que
ces populations étaient structurées de façon non triviales. Ce 
résultat très général a été vérifié à de nombreuses reprises, que ce soit en
laboratoire, comme chez la drosophile \autocites{madalena1974a} et les acariens
par exemple \autocites{benton2005a}, ou dans des populations naturelles telles que les populations de
moutons de Soay \autocites{coulson2001a,ozgul2009a} ou de cerfs élaphe \autocites{langvatn1999a}. 
Cela implique que la description d'une population comme un tout ou comme
un assemblage de classes crées artificiellement représente généralement mal la
réalité et n'intègre pas suffisamment de complexité pour décrire fidèlement les
mécanismes qui régulent sa dynamique.

\subsection{Différents niveaux de structuration}

Dans une population, la structure émerge de l'hétérogénéité entre les
individus d'une même espèce \autocites{benton2006a}. Plusieurs formes de
structuration ont été classiquement prises en compte en écologie.

\subsubsection{Structuration spatiale}

Une première forme de structuration évidente est la structuration spatiale.
Celle-ci décrit comment les individus d'une population s'organisent dans
l'espace, et ce faisant, modifient leurs interactions entre eux et avec leur environnement. 

La structuration spatiale des populations répond souvent à l'hétérogénéité de
leur habitat. Ces hétérogénéités ont des conséquences directes sur la dynamiques
des populations, par exemple en modifiant les schémas de dispersion des
individus \autocites{hiebeler2000a}, leur fitness \autocites{zajkac2008a},
l'accès aux ressources \autocites{burger2008a}, la sensibilité aux parasite ou
pathogènes \autocites{su2009a}, \textit{etc}.

L'étude de la dynamique des populations structurées spatialement constitue un
champs de recherche extrêmement large et varié auquel notre étude ne se rattache
pas directement. 

\subsubsection{Structuration génétique}

Conséquence de la structuration spatiale, les populations sont souvent également
structurées génétiquement. Les individus spatialement les plus proches les uns
des autres, notamment dans des méta-populations, sont également plus proches
génétiquement. L'analyse génétique d'une population permet alors d'obtenir des
informations sur ses origines et sa structuration spatiale
\autocites{repaci2006a,booth2009a,jorde2007a}.

\subsubsection{Structuration en stades}

Une des causes principales de l'hétérogénéité à l'origine de la
structuration des populations vient du cycle de vie des individus. Lorsque le
cycle de vie d'une espèce est tel que les traits d'histoire de vie comme la
croissance, la reproduction ou la mortalité varient beaucoup entre des étapes
différentes mais sont très similaires au sein d'une même étape, on peut alors
séparer la population en plusieurs stades définis par les différentes étapes du
cycle de vie.

Cette forme de structuration, intégrée dans des modèles de dynamique de
population depuis une trentaine d'année \autocites{gurney1983a,nisbet1983a},
permet une description plus rigoureuse des relations entre les traits
d'histoire de vie individuels et la dynamique de la population. Cette forme de
structuration et les modèles qui en découlent ont principalement été appliqués à
des populations d'invertébrés et d'insectes dont le cycle de vie contient un ou
plusieurs événements de métamorphose
\autocites{gurney1980a,gurney1983a,nisbet1983a,nisbet1989a,mccauley1996a}.

\subsubsection{Structuration en âges}

La structuration en âge d'une population est également couramment utilisée en
dynamique des populations lorsque l'âge de l'individu devient l'unité pertinente
pour suivre les variations des traits d'histoire de vie. L'âge des individus est
maintenant très couramment incorporé lors des études de dynamique de populations
naturelles ou théoriques
\autocite[par
ex. ][]{coulson2008a,marteinsdottir2002a,worden2010a,robinson2013a}. 
Cependant, considérer une structuration par l'âge uniquement oblige à fixer pour
tous les individus une même progression dans les trajectoires de vie. Or, il
peut exister des différences d'histoire de vie entre deux individus du même âge
dans une même population. 

\subsubsection{Structuration physiologique}

Afin de palier à ce défaut, les écologues ont considéré des caractères
physiologiques comme éléments structurants des populations. De cette façon, les
traits d'histoire de vie des individus n'ont pas besoin d'être divisibles en des
classes bien distinctes, mais l'impact de l'état
physiologique de l'individu sur ses traits d'histoire de vie -- que sont par
exemple la reproduction, la croissance, la mortalité ou la vitesse d'ingestion
de l'énergie -- est tout de même pris en compte. 

Un cadre théorique complet a été développé pour permettre d'étudier la dynamique
des populations structurées physiologiquement. Les modèles de
population physiologiquement structurés (modèles PSP pour ``Physiologically
Structured Population'') constituent une part importante de ce cadre théorique
\autocites{metz1986a,de-roos1992a,de-roos1997a}, et permettent de tenir compte
des histoires de vie dans les quelles les traits physiologiques et les
interactions écologiques varient de façon continue. 

Une sous partie des modèles PSP s'intéressent particulièrement au rôle de la
taille corporelle dans les interactions écologiques, l'histoire de vie, et les
répercutions sur la dynamique des populations. 

\subsection{L'importance de la taille corporelle}

La taille corporelle constitue un facteur clé dans la compréhension des rapports
entre état individuel et traits d'histoire de vie, et leur conséquences sur la
dynamique des populations et des communautés. 

\subsubsection{Impact sur les traits d'histoire de vie}

L'influence de la taille corporelle sur les performances écologiques, mesurées
notamment par les taux vitaux (croissance, reproduction ou mortalité) ou les
interactions trophiques, on fait l'objet d'un grand nombre d'études théoriques
et expérimentales
\autocites[][\ldots]{peters1986a,calder1996a,de-roos2001a,claessen2004a}. Par
exemple, des individus plus larges vont généralement être plus efficaces dans
leur recherche de nourriture, pouvoir se nourrir de proies plus grandes et
courir un risque réduit de prédation comparé à des individus plus petits
\autocites{paradis1996a}.
Dans un autre registre, les capacités de recherche de nourriture de la daphnée
en fonction de sa taille ont été mesurées en détail dans un grand nombre de
conditions différentes, et reliées à l'allocation de l'énergie assimilée à la
croissance, à la reproduction ou au métabolisme \autocites[par ex.
][]{lampert1978a,gurney1990a,mccauley1990a,kooijman2000a}. De nombreuses études
se sont également intéressées au comportement de recherche de nourriture et à la
gestion de l'énergie chez des populations de poissons \autocites[par ex.
][]{elliott1975a,mittelbach1981a,fuiman1994a,hjelm2001a}. Ces différentes études
expérimentales ont conduit au développement de modèles génériques reliant
énergie et traits d'histoire de vie tels que la capacité à rechercher de la
nourriture, la croissance et le développement
\autocites{kooijman2000a, nisbet2000a, west2001a}. Dans ses travaux,
\textcite{kooijman2000a} propose un cadre théorique à la fois concis et complet,
le budget énergétique dynamique (``dynamic energy budget''), qui décrit la
consommation  de l'énergie et des nutriments à l'échelle de l'individu en
relation avec sa taille corporelle, et leur utilisation pour les différents
traits d'histoire de vie. C'est dans ce cadre théorique notamment que l'on
étudie l'impact sur la dynamique des populations de la dépendance des traits
d'histoire de vie à la taille corporelle.

\subsubsection{Interaction histoire de vie et la dynamique des populations}
\label{modelPopStru}
Les premier modèles de dynamique des populations structurées par stade
\autocites{gurney1980a,gurney1983a,nisbet1983a,lawton1981a} ont été inspirés par
des observations de populations d'insectes, naturelles ou en laboratoire. Ces
populations montraient une dynamique fluctuante, même si l'environnement pouvait
être considéré comme constant
\autocites{nicholson1954a,gurney1983a,ebenman1988a,godfray1989a}. Ces dynamiques
fluctuantes présentaient la particularité d'être due à une succession dans le
temps de générations sans chevauchement, même si les
histoires de vie individuelles le rendaient possible. Ces études ainsi que les
travaux de \textcite{gurney1985a} ont alors permis d'identifier deux types de
cyclicité différentes suivant la période du cycle, par rapport au temps de
génération
\begin{enumerate*}[label=(\roman*), before=\unskip{ : }, itemjoin={{ ; }},
itemjoin*={{ ; et }}]
  \item des cycles d'une seule génération (``single generation cycles'') avec une
  périodicité autour du temps de génération, éventuellement légèrement
  supérieure, mais toujours inférieure à deux fois le temps de génération
  \item des cycles dit ``delayed-feedback cycles'' où la périodicité est cette
  fois entre deux et quatre fois le temps de génération. 
\end{enumerate*} 
Ces cycles de différentes périodes sont expliqués par une compétition
différentielle entre les différents stades présents dans la population, et à un
changement des taux vitaux individuels avec la densité d'individus dans les
chaque stade.

L'identification de ces deux types de cycles a été une avancée majeure dans la
théorie des interactions entre histoire de vie et dynamique des populations. Ces
deux concepts sont généraux et s'appliquent plus largement qu'aux seuls modèles
structurés par stades. Ils se retrouvent notamment dans des modèles
structurés en âge. Ces cycles ont par la suite fait l'objet d'études expérimentales.
\textcite{mccauley1987a} ont par exemple observé l'existence de cycles ``single
generation'' dans le système ressources -- consommateur constitué d'algues et de
daphnées. Plus récemment, \textcite{murdoch2002a} ont démontré l'importance des
deux types de cycles dans les populations naturelles en citant plus de cent
espèces différentes montrant des dynamiques cycliques ressemblantes. Ceci a
permis de montrer qu'un grand nombre de dynamiques de populations observées dans
la nature sont en partie expliquées par des aspects individuels liés aux
histoires de vie et à la structure des populations.

L'existence de ces cycles est principalement du au temps nécessaire à
un juvénile pour atteindre la maturité, appelé ``juvenile delay''.
\textcite{de-roos1990a} et \textcite{de-roos1997a} ont montré, en modélisant une
population de daphnées se nourrissant d'algues, que les cycles de génération
apparaissaient à cause des changement dans la relation âge--taille avec
le niveau de ressources. L'augmentation du niveau de ressource a plusieurs
effets \begin{enumerate*}[label=(\roman*), before=\unskip{ : }, itemjoin={{ ; }},
itemjoin*={{ ; et }}]\item les individus se développent plus vite et maturent
donc plus tôt \item ils atteignent une plus grande taille et sont plus efficace
dans leur recherche de nourriture \item possèdent une plus grande
fécondité.\end{enumerate*} La variation dans l'âge à maturité est un facteur
majeur de la déstabilisation en cycles de génération.

\label{competPopStru}L'étude plus approfondie de modèles PSP et de modèles à
deux stades \autocite[juvéniles et adultes, ][]{de-roos2003a} a également montré que les
cycles de génération apparaissaient dans le cas d'un déséquilibre de compétition
entre des individus de taille différente. Si les individus les plus petits sont
compétitivement supérieurs, la dynamique de la population tend vers des cycles
de génération dominés par la présence des juvéniles, appelés ``juvenile driven
cycles''. A l'inverse, lorsque l'avantage compétitif est aux plus grands
individus, les caractéristiques des cycles observés changent.
En particulier, l'amplitude diminue, la fécondité et le nombre d'adulte
n'oscillent plus en phase, et la survie des adultes est grandement allongée. Ces
cycles, dominés par la présence des adultes, sont appelés ``adult driven cycles''.
Si les capacité de compétition sont équilibrées entre les individus de
différentes tailles, les oscillations disparaissent et la dynamique se
stabilise. 

Toutefois, ces modèles n'ont jusqu'à présent pas permis d'expliquer
les cycles dit ``delayed-feedback cycles'' qui ne sont présents que dans les
modèles structurés en stades dans lesquels un effet différé de la compétition
intra-stade sur les performances écologiques est explicitement incorporé.

Bien que les cycles de générations soit présents dans un grand nombre de modèles
de populations structurées, certains modèles plus complexes développent de
nouvelles dynamiques. Par exemple, l'ajout de cannibalisme intra-stade permet de
prédire des dynamiques fluctuantes apériodiques, voir chaotiques
\autocites{costantino1997a,dennis1997a}.

\subsubsection{Impact sur la structure et la dynamique des communautés}

L'incorporation de la structure des populations dans la compréhension de leurs 
dynamiques a aussi un impact direct sur la compréhension de certaines dynamiques
de communautés. Si l'on considère une population de proie structurée en âge,
taille ou stade, il est possible d'imaginer que des prédateurs se
spécialisant sur des étapes différentes de la vie des proies 
occuperaient des niches écologiques différentes, et pourraient ainsi coexister dans une même
communauté. 

Une conséquence nouvelle de la prédation spécialisée sur une classe de taille
précise a été montrée par \textcite{de-roos2002a} en modélisant une chaine
trophique linéaire à trois niveaux \begin{enumerate*}[label=(\roman*),
before=\unskip{ : }, itemjoin={{ ; }}, itemjoin*={{ ; et }}]\item une ressource
non structurée \item un consommateur structuré en taille \item un prédateur non
structuré qui s'attaque aux consommateurs de petite taille.\end{enumerate*}
Alors que le modèle classique non structuré prédit une corrélation positive
entre la densité de prédateur et celle de la ressource dès lors que le prédateur
peut se maintenir, le modèle structuré prédit une bistabilité entre un équilibre
sans prédateur et un équilibre avec prédateur pour une grande région de
productivité de la ressource. Cette bistabilité est due aux changements que le
prédateur cause dans la distribution en taille des proies. En s'attaquant aux
plus petits individus, le prédateur relâche la pression de compétition subie par
les plus grands consommateurs, ce qui leur permet de grandir et de se reproduire
d'avantage.
A son tour, cela augmente la disponibilité en proies vulnérables aux prédateurs.
Ainsi, les prédateurs montrent alors un effet Allee émergent alors même qu'ils
ne possèdent aucune des caractéristiques classiquement requises telles que la
recherche de nourriture en groupe ou la reproduction sexuelle. Cet effet Allee
émergent n'est en revanche possible que si, en l'absence de prédateurs, la
croissance et le développement des juvéniles chez les proies dépendent de leur
densité \autocites{de-roos2003a}. Dans ce cas de figure, dans la zone de
bistabilité, le prédateur ne peut s'établir que s'il est présent en densité
suffisante pour impulser un changement durable dans la distribution en taille de
la population de consommateur, nécessaire à sa propre subsistance.

En conséquence de cet effet Allee émergent, le prédateur est susceptible de
subir un effondrement de sa population si la productivité du système passe sous
un seuil critique. Passé ce seuil, le prédateur ne pourra plus se réinstaller
dans la communauté, même si la productivité repasse le seuil en question. 
Cet effet Allee émergent ainsi que ses conséquences sur les population de
prédateurs sont probablement relativement communs dans les populations
naturelles, en particulier chez les daphnées \autocites{mccauley1987a} ou certaines
populations de poissons, et pourraient expliquer la disparition de certaine
populations de prédateurs sans observer leur retour, comme pour la morue dans le
nord-ouest de l'Atlantique \autocites{carscadden2001a}.

Ainsi, contrairement aux modèles classiques de réseaux  trophiques où les
conséquences de la consommation d'un niveau trophique sont toujours négatives,
les résultats liés aux populations structurées montrent qu'à cause de la
dépendance des performances écologiques aux histoires de vies et à la taille des
individus, les rétro-actions des individus sur leurs propres performances
peuvent être subtiles, et donner lieu à des phénomènes nouveaux en écologie des
populations et des communautés. 

