\chaptermark{Problématiques}
\chapter{Problématiques étudiées}

Dans un contexte de changements globaux affectant l'ensemble des écosystèmes et
leurs populations, et en particulier de changement climatique, il est
essentiel de décrypter et de comprendre correctement les dynamiques des
populations pour pouvoir les prédire le plus fidèlement possible et anticiper
les changement quelles sont susceptibles de subir.

Nous avons pu voir au cours de ce chapitre que plusieurs éléments entrent
en jeu quand il s'agit de décrire précisément la dynamique d'une population
naturelle. En particulier, la taille corporelle des individus joue un rôle
essentiel à tous les niveaux de régulation de la population. En effet, la taille
de chacun des individus d'une population constitue la structure en taille de la
population, et cette structure est un élément essentiel dans la
dynamique des populations. La taille des individus détermine fortement
leur capacité à accéder et gérer les ressources et l'énergie acquise, ce qui
influe sur la dynamique de la population. Prendre en compte la structure de la
population dans la modélisation de sa dynamique permet de prédire des dynamiques
observées dans la nature, qui ne sont pas expliquées par les modèles classiques
non structurés.

De plus, la taille des individus a un impact direct sur le résultat de leurs
interactions, que ce soit au sein d'une population ou entre différentes
populations. Les interactions entre individus interviennent en particulier dans
la compétition par interférence qui influe sur les performances écologiques des
individus et donc sur la dynamique de la population. 

Enfin, la taille individuelle est déterminante dans la réponse à la température
et à ses changements, puisque la température influe directement sur la taille
atteinte à un âge donnée. A son tour, cela impacte directement les performances
individuelles, la structure de la population et la composition de la communauté. 

Au vu de l'importance de la taille individuelle et de la structure de la
population dans la régulation de la dynamique des populations via les
interactions entre individus et la réponse à l'environnement, je me suis
intéressé dans cette thèse au rôle que joue la compétition par interférence dans
la régulation des populations structurées, à la fois d'un point de vu empirique
et théorique, d'abord à une seule condition de température, puis à plusieurs
températures pour comprendre comment la température modifie les effets de la
compétition par interférence. Afin de répondre à ces différents points, j'ai
séparé mon étude en deux questions principales :
\begin{enumerate}
  \item Quelle est le rôle de la compétition par interférence dans la régulation
  des populations structurées en taille ?
  \item Comment la température interagit avec les mécanismes de compétition et
  modifie leurs impacts sur la régulation des populations structurées ?
\end{enumerate}

\section{Le rôle de la compétition par interférence dans la dynamique des
populations structurées}

Comme nous l'avons dit, la structure de la population et les différences de
performances des individus de différentes taille est à l'origine d'un certains
nombre de dynamiques que les modèles classique de dynamique des populations ne
peuvent pas expliquer. La compétition par exploitation a déjà été largement
étudiée, mais peu d'études empiriques ou théoriques se sont attelées au rôle de
la compétition par interférence dans la dynamiques de populations, et
particulièrement dans la dynamiques des populations structurées.

\subsection{Compétition par interférence et populations structurées,
analyse de séries temporelles de populations expérimentales (Chapitre \ref{chap:sp})}

Dans un premier temps, nous nous sommes intéressé aux effets de la compétition
par interférence dans la régulation des populations structurées d'un point de vu
empirique. Au cours de cette étude, nous avons suivi pendant plusieurs années
des populations de collemboles \textit{Folsomia candida} dans un environnement
contrôlé. Le collembole constitue en effet une espèce modèle en écologie
\autocites{fountain2005a} qui permet le suivi fin et précis de nombreuses
population avec assez peu de difficultés. Des mesures hebdomadaires de la
structure détaillée des populations nous ont permis d'étudier le rôle des
structures passées et présentes des populations dans la régulation de la
dynamique future, ainsi que l'importance de la présence dans les populations
d'individus de grande taille, particulièrement performants dans la compétition par interférence, dans les
dynamiques observées. 

\subsection{Analyse théorique du rôle de la compétition par interférence dans
la dynamique des populations structurées (Chapitre \ref{chap:amnat})}

D'un point de vu théorique, les conséquences de la compétition par interférence
sur la dynamique des population structurées n'ont encore pas été explorées. Dans
une seconde étude, nous proposons une extension du modèle classique de
\textcites{kooijman1984a} en ajoutant explicitement de la compétition par
interférence. Nous définissons l'interférence comme une interactions directe
entre deux individus où le plus grand des deux a un avantage compétitif qui
réduit les possibilités du plus petit d'accéder à la ressource. Nous étudions
différentes conditions allant d'une compétition par exploitation pure à une
domination de la compétition par interférence. Notre objectif est d'étudier le
rôle de la compétition intra-spécifique par interférence sur la dynamique des
populations structurées en utilisant les effets largement étudiés de la
compétition par exploitation comme référence. Nous comparons ensuite les
prédiction de notre modèle aux résultats tirés de l'analyse des séries
temporelles issues des populations expérimentales de l'étude précédente.

\subsection{Vérification expérimentale de l'importance des individus de
grande taille et de la compétition par interférence dans la dynamique des
populations structurées (Chapitre \ref{chap:sm})}

Nos observations expérimentales et notre étude théorique semblent confirmer le
rôle prépondérant que jouent les individus de grande taille dans la dynamique de
la structure d'une population de collemboles. Afin de vérifier expérimentalement
les conclusions des précédentes études et de déterminer le rôle exacte des
individus de grande taille dans la structuration de nos populations, nous avons
réalisé une seconde étude expérimentale. Au cours de cette étude, nous avons
utiliser des populations arrivées à l'équilibre présentant des structures
similaires, et nous avons alors isolé certaines classes d'individus de ces
populations et fondé avec chacune de nouvelles populations. Nous avons ensuite
observer le retour à l'équilibre de chaque sous population après perturbation.
Parallèlement à cette analyse sur le long terme, nous avons réalisé des suivis
comportementaux au niveau de la ressource afin d'observer qui accède à la
nourriture et dans quelles conditions. Cette étude permet de répondre à deux
questions \begin{enumerate*}[label=(\roman*), before=\unskip{ : }, itemjoin={{ ? }},
itemjoin*={{ ? Et }}] \item quel rôle jouent les individus de grande taille dans
l'établissement d'une structure cyclique ou stable dans nos populations de
collemboles \item L'accès à la nourriture se fait-il de façon homogène ou
peut-on observer des comportement d'interférence dans l'accès aux ressources ?
\end{enumerate*}

\section{Interaction entre température et compétition par interférence dans
la régulation des populations structurées (Chapitre \ref{chap:fip})}

Nous avons pu voir que la température est un élément extérieur essentiel dans la
régulation physiologique des individus, et a des répercussion immédiates sur les
dynamiques des populations et des communautés. 

Ayant démontré l'importance de la compétition par interférence dans la
régulation de la dynamique des populations structurées, nous nous intéressons
maintenant aux interactions entre effet de température et processus individuels
dans la dynamique des populations structurées. Les approches classiques dans
l'étude de l'influence de la température consistent à mesurer des normes de
réaction à la température sur des individus isolés ou des petites cohortes. Mais
les connaissance acquises par ces études ne permettent pas d'extrapoler
directement au niveau de la population car il manque les processus de densité
dépendance. Nous nous demandons donc si les normes de réactions individuelles
peuvent aider dans la prédiction de la réponse des populations à un changement
de température. Nous cherchons à comprendre comment les effets de la température
sur les traits d'histoire de vie sont modulés par les rétroactions
démographiques.

En comparant des normes de réactions mesurées sur des individus isolées à leur
équivalents mesurées dans des populations, nous cherchons à extraire les effets
de la température sur les traits d'histoire de vie au niveau individuel pour
voir comment la température affecte les poids relatifs de la compétition par
interférence et par exploitation dans la régulation de la population et de sa
structure en taille.

