\sectionmark{Densité dépendance et compétition}
\section{Densité dépendance,
compétition et régulation des populations}
\sectionmark{Densité dépendance et compétition}

\subsection{Densité dépendance}

Les organismes grandissent, se reproduisent puis meurent; ils se développent
dans un environnement donné, et sont affectés par les ressources à leur
disposition. Pendant toute ou partie de leur vie, ils sont
entourés d'autres individus de leur propre espèce pour constituer ce que l'on
appelle une population \autocites{begon2009a}. 

Une population évolue dans son écosystème à une échelle géographique finie, ce
qui la soumet à sa propre densité. On appelle alors densité
dépendance le principe qui décrit comment les taux intrinsèques de la
population -- tels que le taux d'accroissement, les taux de naissance ou de
mort, les taux d'immigration ou d'émigration, \textit{etc.} -- varient à cause
de la taille de la population elle même, ou de sa densité.

\subsubsection{Principes généraux et définition}

Formulée simplement, la densité dépendance représente l'idée que les
comportements ou les traits écologiques varient en fonction du nombre
d'individus présents dans la population. Ces traits écologiques comprennent
classiquement le taux de croissance de la population ainsi que les principaux
taux démographiques de la population (naissance, mort, immigration et
émigration), mais peuvent également se référer aux taux de croissance
individuels, aux taux de fécondité, ou à d'autres taux ou comportements au niveau individuel
\autocites{royama1977a}.

Le principe de densité dépendance, ou \textbf{densité dépendance directe},
impose un effet négatif sur les taux responsables de l'accroissement de la population, et
un effet positif sur les taux responsable de sa décroissance (par opposition à
la densité indirecte ou effet Allee qui a l'effet inverse). Si l'on note $N$ la
densité d'individus dans une population, alors une augmentation de $N$
entraînera une diminution des taux tels que le taux de fécondité ou le taux de
croissance, le taux de naissance dans la population ou d'immigration, et par
incidence, du taux d'accroissement de la population. A l'inverse,
l'augementation de $N$ provoque l'augmentation du taux de mortalité ou
d'émigration \autocites{hixon2009a}. Dans le cas ou le paramètre comportemental
ou écologique à l'étude ne varie pas avec $N$ il est alors dit densité
indépendant.

Le principe de la densité dépendance est un élément fondamental, que ce soit en
écologie et biologie des populations \autocites{kingsland1995a}, en pêcherie
\autocites{rose2001a}, en gestion de la biodiversité et de la vie sauvage
\autocites{gordon2004a}, dans le contrôle des ravageurs \autocites{walde1988a},
ou en biologie de la conservation \autocites{ginzburg1990a}. En effet, ce mécanisme est
essentiel dans la régulation des populations \autocites{murdoch1994a,
turchin1990a}. Le principe de la densité dépendance dans le contexte de la
régulation des populations a été modélisé pour la première fois par
\textcite{verhulst1838a} et a été très largement réutilisé et adapté depuis dans
de nombreux modèles. 

Une densité dépendance positive peut également se manifester sous la forme de
l'effet Allee \autocites{allee1931animal,courchamp1999a}. Dans ce cas, il existe
un niveau minimum de densité que doit avoir la population pour être viable. En dessous de
ce minimum, la dynamique de la population tendra inexorablement vers
l'extinction. Ce phénomène se produit par exemple lorsqu'avec le déclin de la
population, les rencontres entre partenaires sexuels sont de plus en plus
difficiles, et les individus ne parviennent plus à se reproduire en nombre
suffisant pour permettre à la population de se maintenir. Bien qu'étant
relativement répandu, nous n'accorderont pas ici plus de place à ce mécanisme. Nous nous intéresserons dans
la suite uniquement à la densité dépendance négative, ou densité dépendance
directe. 

\subsubsection{Les mécanismes de la densité dépendance}

Les causes directes de la densité dépendance sont en premier lieu la
compétition, et dans certains cas la prédation (incluant le parasitisme et les
maladies). Par définition, la compétition est densité dépendante puisqu'elle
relie le nombre d'individu à la disponibilité d'une ressource donnée. Ainsi, la
compétition pour un territoire, pour un refuge pour se protéger contre des
conditions environnementales hostiles ou contre un prédateur, pour la
nourriture, ou pour la possibilité de se reproduire peuvent toutes être à
l'origine d'une réponse densité dépendante des taux démographiques d'une population
\autocites{keddy1989a,begon2009a}.

De la même façon, les prédateurs peuvent également causer de la densité
dépendance chez les proies, notamment sur la mortalité, par différents
mécanismes \autocites{taylor1984a} 
\begin{enumerate*}[label=(\roman*),
before=\unskip{ : }, itemjoin={{ ; }}, itemjoin*={{ ; et }}] 
\item si la
population de prédateurs réagit suffisamment vite à la présence de proies,
l'augmentation de la population de proies provoque l'augmentation de la
mortalité par prédation 
\item la configuration spatiale de l'environnement peut être telle
que de nombreux prédateurs se retrouvent dans un même endroit si les proies y
sont nombreuses, imposant alors une forte mortalité 
\item une réponse
développementale du prédateur peut entraîner une augmentation du taux de
consommation du prédateur lorsque les proies sont plus abondantes 
\item une réponse fonctionnelle du prédateur de type III
\autocites{holling1965a} cause une mortalité densité dépendante chez la proie lorsqu'elle est en faible
densité.
\end{enumerate*}

Dans la suite des travaux, nous laisserons de côté les mécanismes de régulation
par la prédation pour nous intéresser exclusivement à la régulation liée à la
compétition pour les ressources. 

\subsection{La compétition par exploitation}

La compétition pour les ressources est l'une des interactions écologiques
essentielles dans la régulation des populations et des communautés. Elle est définie comme une
interaction entre organismes telle que les performances d'un individus en termes
de fécondité, croissance ou survie, sont réduites par la présence d'autres
organismes \autocites{volterra1931a, gause1932a, park1948a, park1954a,
park1957a}.
La compétition peut intervenir aussi bien entre des individus d'espèces
différentes (compétition inter-spécifique) qu'au sein d'une population
d'individus de la même espèce (compétition intra-spécifique). 
Il existe dans la nature deux grands types de compétition
\begin{enumerate*}[label=(\roman*), before=\unskip{ : }, itemjoin={{ ; }},
itemjoin*={{ ; et }}] \item la compétition par exploitation \item la compétition
par interférence \end{enumerate*} \autocites{park1954a, park1962a, begon2009a}.

\subsubsection{Définition}


La compétition par exploitation est une forme de compétition où les
individus ont un effet négatif les uns sur les autres en consommant une
ressource qui leur est commune \autocites{goss-custard1980a,
vance1984a, begon2009a}. Les individus concernés n'interagissent alors pas directement les
uns avec les autres, ils sont sensibles au niveau de ressources
disponible après consommation par d'autres individus. Cette compétition est
donc dite indirecte car elle ne requiert pas de contact physique ou
d'interaction directe entre les individus pour entrer en jeu. Enfin, il est
indispensable que la ressource considérée soit limitante pour que les individus
entrent en compétition \autocites{begon2009a}. 

\subsubsection{Dans les modèles de populations non structurées}

En écologie des populations, la compétition par interférence a été introduite 
dans les premiers modèles par la fonction logistique
\autocites{verhulst1838a}, sous la forme d'une capacité de charge (classiquement notée $K$). La variation
de la densité de la population $N$ s'écrit alors sous la forme
$$\frac{dN}{dt}=rN \left(1-\frac{N}{K}\right)$$ où $r$ est le taux
d'accroissement de la population. On constate alors que le taux de croissance
\textit{per capita} de la population $\frac{dN}{dt}\cdot \frac{1}{N}$ suit alors
une loi affine décroissante dont la pente est $-\frac{1}{K}$. En d'autres termes, le
taux de croissance de la population tend vers 0 lorsque la densité de la
population s'approche de $K$. De plus, si la population est moins dense que $K$,
elle va croître jusqu'à atteindre sa capacité de charge, mais à l'inverse, si
elle est plus dense que $K$, le taux de croissance de la population est négatif
et la densité va décroître jusqu'à $K$.

Ce modèle de dynamique de population intégrant de la densité dépendance fut un
des premiers modèles présentés, mais il existe depuis un très grand nombre de
déclinaisons ou d'alternatives à ce modèle. On peut citer par exemple les
modèles à reproduction discrète où la compétition a été intégrée sous la forme
de la loi de Beverton Holt. Dans ces différents modèles, la compétition est
représentée sous une forme symétrique, sans aucune différence entre les
individus constituants la population. 

\subsubsection{En dynamique des populations structurées}

L'aspect symétrique de la compétition telle que décrite précédemment représente
bien les comportements moyens d'une population. Cependant, il est aisé
d'imaginer que tous les individus d'une population ne sont pas identiques, et
donc pas égaux non plus face à la compétition. 

Si les différences entre individus sont fortement marquées, ou influent
beaucoup sur leurs performances individuelles et écologiques, il devient alors
nécessaire de considérer la structure de la population lorsque l'on cherche à
décrire sa dynamique. Nous avons déjà fait référence aux conséquences de la
compétition par exploitation sur la dynamique d'une population structurée en
taille (cf. section~\ref{modelPopStru} page~\pageref{modelPopStru}). L'étude
des modèles physiologiquement structurés a montré que des capacités compétitives différentielles selon l'état
physiologique de l'individu avaient un impact très fort sur la dynamique que
suivait la population. Dans un modèle simplifié en deux stades aux capacités de
compétition différentes, juvéniles et adultes, un avantage compétitif aux
juvéniles entraînait des cycles de génération dit ``\textbf{juvenile driven}'',
alors qu'un avantage aux adultes conduisait à des cycles ``\textbf{adult
driven}'' aux caractéristiques différentes \autocites{de-roos2003a}. Une
compétition équilibrée se traduit par une dynamique stable de la population dans son ensemble. 


\subsection{La compétition par interférence}

\subsubsection{Définition}

A l'opposé de la compétition par exploitation, il existe une autre forme de
compétition appelée compétition par interférence. Cette compétition intervient
quand les individus subissent une interaction directe négative où l'un d'eux
réduit la capacité de l'autre à exploiter une ressource commune, quel que soit
le niveau de cette ressource \autocites{park1954a,vance1984a}. Ces interactions
peuvent prendre différentes formes : agressivité \autocites{schoener1976a},
territorialité \autocites{walls1990a,kennedy1996a}, allelopathie
\autocites{harper1977a,rice1984a,nilsson1994a}, surdéveloppement et prolifération
\autocites{connell1961a,paine1966a}, \ldots~Par définition dans la compétition
par interférence, le compétiteur le plus fort réduit les performances du
compétiteur le plus faible en lui interdisant
partiellement ou totalement l'accès à la ressource convoitée
\autocites{schoener1983a, thompson1993a}. De fait, la domination dans une
interaction par interférence est donc souvent liée aux différences
physiologiques entre les individus, et notamment, souvent aux différences de
taille corporelle, auquel cas, le plus grand est généralement le plus
compétitif \autocites{mccormick2012a}. Dans le cas de la compétition
intra-spécifique, les conséquences de la compétition par interférence sur la
dynamique de la population dépendent donc directement de la distribution en
taille des individus de la population, ainsi que de leurs traits d'histoire de
vie. 

\subsubsection{Modèles de compétition par interférence}

La compétition par interférence a été largement observée et décrite dans la
nature, que ce soit dans les cas inter-spécifiques ou intra-spécifiques.
Cependant, les tentatives d'incorporer la compétition par interférence dans les
modèles de dynamique de populations sont encore relativement rares. De plus, la
plupart de ces études se concentrent sur la compétition inter-spécifique
\autocites{case1974a, carothers1984a, vance1984a, adler2000a}. Une version de la
compétition par interférence a notamment été proposée par Arditi et Ginzburg
dans leur modèle ratio-dépendant \autocites{arditi1989a,arditi2012a,arditi1991a}.
Dans ce modèle de dynamique de populations dans un système prédateur-proie, le
taux annuel de consommation de la proie par le prédateur dépend du nombre de
proies présentes par prédateur, plutôt que du nombre absolu de proie dans le
système \autocite[voir][pour les détails et dérivations du modèle]{arditi2012a}.
Ce modèle dit ``ratio dépendant'' conduit à des dynamiques différentes de ce qui
est attendu dans le modèle de comparaison pour la compétition par exploitation,
à savoir le modèle Rosenzweig-MacArthur, dans lequel le taux d'attaque du
prédateur dépend uniquement de la densité de proie. Par expemple, le paradoxe de
l'enrichissement qui conduit sous certaines conditions à un accroissement de la
population de proies lorsque les prédateurs augmentent en nombre, est absent du
modèle ratio-dépendant \autocites{arditi2012a}. 

D'autres études proposent des approches différentes. Par exemple,
\textcite{amarasekare2002a} propose un modèle réunissant compétition par
exploitation et par interférence dans lequel la dynamique de la ressource est
décrite explicitement. Avec ce modèle, l'auteur étudie la possibilité de la
coexistence de deux espèces en compétition pour une même ressource. Cette étude
montre alors deux cas de figure contrastés \begin{enumerate*}[label=(\roman*),
before=\unskip{ : }, itemjoin={{ ; }}, itemjoin*={{ ; et }}] \item si la
compétition par interférence a un coût pour les deux compétiteurs, les deux
espèces ne peuvent pas cohabiter, même si l'espèce dominée dans la compétition
par exploitation est dominante dans la compétition par interférence \item si la
compétition par interférence est coûteuse pour le perdant, mais strictement
bénéfique pour le gagnant, les deux espèces peuvent cohabiter si l'infériorité
dans la compétition par exploitation est contrebalancée par une supériorité dans
la compétition par interférence. \end{enumerate*}

La compétition par interférence peut également être considérée du point de vue
intra-spécifique \autocites{walde1984a, crowley1987a, maddonni2004a,
smallegange2006a}. Dans une étude récente, \textcite{de-villemereuil2011a} ont
étudié différentes réponses fonctionnelles pour le consommateur, en étendant les
réponses fonctionnelles classiques pour tenir compte des comportements
d'interférence inter et intra-spécifiques. Dans plusieurs exemples, ils montrent
que la compétition par interférence intra-spécifique a un impact plus
fort sur la régulation de la dynamique des populations étudiées que la compétition par
interférence inter-spécifique. 

\subsubsection{Modéliser la compétition par interférence dans une population
structurée}

Le signe et l'intensité de la compétition par interférence dépend généralement
des différences d'histoire de vie entre les compétiteurs (différences de force,
de sexe, de taille corporelle,\ldots). Pour décrire précisément ses effets sur
la dynamique de la population, il est donc nécessaire d'adopter une approche
tenant compte de la structure de la population. Dans ce cadre, modéliser un système ressource-consommateur simple
incluant de la compétition par interférence chez le consommateur nécessite une
approche centrée sur l'individu. Les modèles de population structurées physiologiquement (cf.
section~\ref{modelPopStru} page~\pageref{modelPopStru}) apportent les éléments
nécessaire à l'étude de la compétition intra-spécifique chez une population de
consommateurs structurée en taille. En effet, ces modèles tiennent compte
explicitement de la distribution en taille de la population et dérivent la
dynamique à l'échelle de la population des processus modélisés à l'échelle de
l'individu, tels que la croissance, la reproduction ou la mortalité
\autocites{kooijman1984a, metz1986a, de-roos1997a}. De plus, puisque ces modèles
intègrent directement le développement ontogénétique des individus, ils rendent
possible l'intégration d'interactions compétitives dépendantes de la taille.

Ces modèles ont déjà été
étudiés dans de nombreuses configurations, et ont donné le jour à une théorie
des conséquences du développement ontogénétique sur la dynamique des populations
et des communautés \autocites{de-roos2012a}. Ce cadre servira de base
à l'étude théorique des conséquences de la compétition par interférence sur la
dynamique d'une population structurée. 
