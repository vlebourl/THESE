\partimage[width=\textwidth]{FigParts/writing3}
\part{Conclusion Générale}

\chapter{Dans une coque de noix}
\chaptermark{Dans une coque de noix}

\section{Principaux résultats de la thèse}

Au cours de cette thèse, nous avons exploré certains détails de la dynamique des
populations structurées, à la fois d'un point de vue expérimental que théorique.
Nous avons porté une attention toute particulière aux rôles que jouent les
interactions taille dépendantes entre individus dans l'établissement de ces
dynamiques, d'abord en observant des populations expérimentales de Collembole
\textit{Folsomia candida} sur le long terme pour en analyser les dynamiques en
fonction de leur structure en taille, puis en modélisant les interactions
tailles dépendantes entre individus dans un modèle physiologiquement structuré,
et enfin en manipulant la structure de la population et en observant les
comportement d'accès aux ressources. Nous nous sommes ensuite intéressé à
l'effet de l'environnement, via la température ambiante, sur ces interactions et
les répercussions sur la dynamique des populations.

Dans ce dernier chapitre, nous reviendrons sur les principaux enseignements que
l'on peut tirer de ces travaux par ses développements tant méthodologiques que
théoriques ou expérimentaux.

\subsection{Les développements méthodologiques}

J'ai choisi de revenir ici sur quelques développements méthodologiques réalisés
au cours de cette thèse car ils n'ont pas seulement occupé une grande partie de
mon travail, mais se sont également révélés indispensable au bon déroulement de
ces travaux. 

\subsubsection{Le phénotypage haut débit}

Le développement de la méthode d'analyse d'image pour le dénombrement et la
mesure de la structure des populations a été un élément clé qui a rendu possible
ce travail. Cette méthode de suivi des populations a été initialisées par
Thomas \textcites{tully2004a} au cours de sa thèse, puis reprise en développée
par François \textcites{mallard2013b} et moi même pendant plusieurs années avant
de faire l'objet d'un chapitre dans un ouvrage du CNRS
\autocites{le-galliard2012a} et d'une publication \autocites{mallard2013a}.

La méthode développé a notamment rendu possible la réplication des expériences
dans une mesure que l'on n'aura pas pu atteindre en dénombrant manuellement les
populations. C'est une méthode fiable qui permet à peu de frais d'accéder à la
structure en taille d'une population (et donc à la densité dans les différentes
classes de taille) en peu de temps et de façon très peu voir pas du tout
intrusive. 

Cette méthode a été développée et appliquée à des populations de collemboles,
mais est susceptible d'être facilement transposée à d'autres systèmes
expérimentaux comme cela a été décrit dans \textcites{mallard2013a}, ce qui lui
donne un grand intérêt pour l'écologie expérimentale et l'étude des traits
d'histoire de vie. 

\subsubsection{Diagrammes structure-temps}

Le travail présenté dans cette thèse n'aurait pas non plus été possible sans le
développement de la représentation graphique en diagrammes structure-temps et
des outils d'analyse graphique associés. En effet, cette technique simple de
représentation des données a permis de rendre cohérentes des données d'une
grande richesse dans les quelles il aurait été facile de se perdre. 

Nous pensons que l'application de cette méthode à des séries temporelles
structurées permettra de mettre en évidence des phénomènes qu'il était jusqu'à
présent difficile de décrire. Cette représentation rempli les critères
d'excellence des représentations graphiques en statistiques décrits par
\textcites{tufte1990a} en condensant une grande quantité d'information sur un
petit espace de représentation, sans déformer les données et en révélant
plusieurs niveaux de détails en une seul fois, des structures fines à la vue
d'ensemble \autocites{tufte2001a}. Avec le développement des méthodes
automatisées telles que celle présentée dans cette thèse \autocites[voir aussi
][]{le-galliard2012a} et la croissance toujours accélérée des bases de données
(``big data''), ce type de méthode devrait à l'avenir se démocratiser, autant en
écologie des populations que dans des domaines plus variés comme en
démographie humaine, en épidémiologie, dans les enquêtes d'opinion, d'usage,
\textit{etc}.

\subsection{Taille corporelle et populations structurées}

Au delà des apports méthodologiques, les travaux présentés dans cette thèse
répondent à des questions fondamentales en écologie des populations quant aux
rôle des interactions entre individus au sein d'une population dans la
régulation de sa structure et de sa dynamique, et à la place qu'occupe la taille
corporelle des individus dans la détermination de ces interactions.

\subsubsection{Taille corporelle et compétition par interférence}

Un premier résultat qui ressort de cet étude est l'importance de la différence
de taille corporelle dans la régulation de la dynamique de la structure en
taille de nos populations expérimentales. Nous avons pu montrer au cours des
différentes expériences menées et suivis de populations que la présence de
classes de tailles différentes en densité différentes avait un impact direct sur
la dynamique que l'on pouvait observer, tant à court terme qu'à long terme. La
présence d'individus de grande taille notamment s'est montré un critère
déterminant dans les dynamiques observées dans les différentes populations. 

En observant plusieurs populations dans les mêmes conditions, nous avons pu
montrer que les différentes structures en taille observées sont en nombre
limité, et peuvent être regroupées en quatre grande catégories. Ces catégories
de structures, que l'on a appelé structures types, se différencies par
l'abondance des juvéniles, la taille des adultes, et leur abondance. Une de ces
catégories est particulièrement remarquable par le fait qu'elle contient trois
modes séparés dans la distribution de la taille (les structures de types 4),
contrairement aux trois autres qui n'en contiennent que deux. Cette structure
type n'a par ailleurs été observée que chez un seul des deux clones étudiés, HA,
et cela quelque soit l'expérience menée. Le point commun entre les quatre
types de structure est la présence d'adultes à une taille corporelle stable
nettement supérieure à la taille à maturité, ce qui est contraire aux
prédictions des modèles de populations structurées régulées par compétition
intra-spécifique par exploitation.

L'observation des dynamiques de court terme dans ces populations a permis de
démontrer que ces adultes de taille relativement élevée jouaient un rôle
déterminant dans les dynamiques de la structure des populations. Leur
disparition progressive ou brutale coïncide de façon quasi systématique avec un
événement de recrutement de juvéniles dans les classes adultes. La perturbation
de la structure des populations dans une seconde expérience a permis de
renforcer le lien de causalité entre la présence des adultes et la croissance ou
non des juvéniles. En effet, retirer l'ensemble des adultes d'une de nos
populations de collemboles provoque un événement massif de recrutement alors que
ne retirer qu'une partie des adultes réduit
grandement le nombre de juvéniles qui parviennent à recruter. Lorsque plusieurs
classes de taille coexistent chez les adultes, le retrait des plus grands, même
moins nombreux, provoque un événement de recrutement plus important que le
retrait des plus petits. Retirer des individus de grande taille permet donc de
rendre la possibilité aux juvéniles de grandir et d'atteindre la maturité.

Cette domination des adultes sur la dynamique de la structure de nos populations
s'explique par la domination qu'ils exercent sur l'accès à la ressource fournie.
Cette hypothèse, issue de l'observation des séries temporelle de la structure
des populations, a été confirmée par l'observation des comportements d'accès à
la ressource et la mesure du biais de taille dans cet accès. Par ces mesures, on
a pu montrer une sur-représentation systématique des individus les plus grands
au contact ou aux abords de la pastille de ressource. Cette pastille de
ressource étant très localisée comparé à l'environnement de vie des Collemboles,
il est alors facile pour la minorité d'individus les plus grands de monopoliser
la ressource et d'en restreindre l'accès aux plus petits. Ces derniers ne
pouvant se nourrir ne peuvent alors pas se développer et restent dans un stade
juvénile jusqu'à ce que le nombre d'adultes diminue suffisamment pour que la
ressource deviennent à nouveau accessible. 

Ce comportement des individus les plus grands repoussant les plus petits aux
abords de la ressource et ses conséquences sur la dynamique de la structure
montrent l'existence d'une compétition par interférence, parfois de forte
intensité, dans nos populations de collemboles. Cette compétition par
interférence contre balance l'avantage énergétique que possèdent les petits
individus et permets ainsi au plus grands de survivre et de dominer la
population. 

\subsubsection{Compétition par interférence et dynamique des populations
structurées}

Nous avons donc montré le rôle de la compétition par interférence dans la
régulation de nos populations structurées en taille, et notamment comment elle
permettait la survie d'individus de grande taille, et parfois de géants très
longévives qui, même en petit nombre, dominent le reste de la population. 

Afin d'étudier plus avant les conséquence de l'existence d'une compétition
intra-spécifique par interférence sur la dynamique d'une population structurée,
nous avons repris le modèle classique de dynamiques de populations
physiologiquement structurées développé pour les Daphnies
\autocites{kooijman1984a}. Nous avons adapté ce modèle à notre système en le
paramétrant pour les collemboles, ce qui a été possible car le cycle de vie des
collemboles présente des similarités avec celui des daphnies dans le sens où les
collemboles ont une croissance continue et sont amétaboles, les juvéniles et les
adultes ayant la même apparence et partageant la même niche, ne se différenciant
que par la taille. Nous avons ensuite étendu le modèle pour y intégrer une
représentation des interactions directes entre individus dépendantes des
tailles respectives des adversaires. Nous avons alors été en mesure de modéliser
différents niveaux de compétition par interférence pour en observer les
conséquences sur la dynamique, comparé à un modèle avec de la
compétition par exploitation seule. Ce modèle présente l'avantage de rester
simple tout en regroupant les aspects principaux de la compétition par
interférence.

Nous avons montré avec ce modèle que la présence de la compétition par
interférence intra-spécifique pouvait avoir différentes conséquences en fonction
de son intensité. En l'absence d'interférence, la théorie existante sur les
modèles PSP décrit déjà les dynamiques prédites, et notamment la présence de
cycles de génération à faible mortalité, et l'effet stabilisant d'une forte
mortalité. Un premier résultat de ce modèle est l'effet stabilisant de la
compétition par interférence lorsqu'elle est à niveau intermédiaire. Si son
intensité est suffisante, la compétition par interférence vient contrebalancer
l'avantage des juvéniles par la compétition par exploitation, et produit ainsi
un effet similaire à l'augmentation de la mortalité basale en stabilisant les
cycles de génération. On obtient alors une structure de population stable
similaire à celle que l'on aurait sans interférence avec une forte mortalité
basale. 

L'augmentation de la compétition par interférence a aussi pour effet de faire
augmenter la capacité des individus les plus grands de la population à accéder à
la ressource. Lorsque cette capacité est suffisante, les individus, qui jusque
là arrêtaient de grandir à une taille proche de la taille à maturité, peuvent
alors reprendre leur croissance, et atteignent des tailles importantes, proche
de leur maximum physiologique possible. Il est intéressant de noter que cela se
produit aussi bien dans une dynamique stable que cyclique. Une fois dépassé le
seuil d'interférence permettant la survie des géants, ceux-ci seront toujours
présents dans les populations et les domineront, à moins que le niveau
d'interférence soit réduit. 

Enfin, à un niveau très élevé, la compétition par interférence donne un tel
avantage aux adultes de grande taille que la dynamique est déstabilisée dans un
nouveau type de cycles. Ces cycles sont également des
cycles de génération, mais de beaucoup plus grande amplitude et plus longue
période. De plus contrairement aux cycles observés en l'absence d'interférence,
ce sont cette fois les adultes qui dominent la dynamique et génèrent les cycles. 

La présence des individus de grande taille et les nouveaux cycles décrits
présentent des similarités avec d'autres phénomènes déjà étudiés dans les
populations structurées, notamment le cannibalisme, ou la compétition
différentielle entre adultes et juvéniles, mais apporte une nouvelle explication
aux dynamiques observées, parfois plus parcimonieuse que celles précédemment
proposées. Ce modèle de dynamique de populations structurées, régulées par la
compétition par interférence, permet donc de compléter les résultats
préexistants sur la compétition par exploitation et le cannibalisme comme
mécanisme régulateur des populations structurées. 

\subsection{Les effets de la température}

Au cours d'une dernière expérience, nous avons comparé des normes de réactions à
la température mesurées sur des individus isolés et dans des populations. Nous
avons pu montrer que les processus de régulation des populations tels que la
compétition par exploitation et par interférence, sont soumis aux
effets de l'environnement et notamment à la température ambiante à laquelle se
trouve la population. 

Nous avons montré en introduction que la température est un facteur abiotique
connu pour son impact sur les individus, les populations et les communautés.
Chez les organismes ectothermes, de part notamment son action sur les réactions
métaboliques, un changement de température affecte toute
la physiologie de l'individu, ce qui se répercute sur sa gestion de l'énergie,
son comportement, sa croissance et son investissement reproducteur. Ces
répercussions en cascades ont ensuite un impact fort sur la dynamique des
populations concernées, et qui elles-mêmes peuvent cascader sur la dynamique de
l'ensemble de la communauté. 

Plus précisément, nous avons montré dans le cas des collemboles que
l'augmentation de la température a un effet prévisible sur les individus isolés,
provoquant notamment une réduction de la taille asymptotique lors d'une
augmentation de température, conformément à la règle taille-température
\autocites{angilletta2009a}. En revanche, dans les populations, l'effet d'une
température plus élevée se retrouve confronté aux mécanismes de compétition
intra-spécifique, et le résultat sur la dynamique des populations et leurs
structures est moins facile à prévoir. Nous avons montré que dans une gamme
intermédiaire de réchauffement, la température provoque une diminution des taux
de croissance et des tailles corporelles adultes, mais dans une bien moindre
mesure que chez les individus isolés, les effets température sont en partie
tamponnés par les mécanismes de densité dépendance. 

De façon plus étonnante, dans des conditions de température élevée
($26\degres$C), nous avons montré que les tailles adultes ne suivent plus la
règle taille-température, même après avoir corrigé les mesures pour les effets
de la densité d'individus. Nous avons pu montré qu'à température élevée, il
devenait plus intéressant pour les individus d'investir dans la croissance et
d'atteindre des tailles supérieures, malgré les désavantages liés à la
température. Nous avons expliqué cela par une inversion des mécanismes de
régulation dans les populations entre compétition par exploitation et
compétition par interférence. Plus précisément, à température basse ou
intermédiaire, la compétition par exploitation pourrait avoir une plus grande
importance que la compétition par interférence dans la régulation des
populations. Nous proposons également que l'intesité des compétitions par
exploitation et interférence augmentent avec la température, mais que la
compétition par interférence augmente plus vite que la compétition par
exploitation, par exemple à cause de l'augmentation de l'activité et du besoin
en nourriture par unité de temps.
Ainsi, à partir d'une température critique, le désavantage d'une grande taille
lié à la température serait sur-compensé par l'avantage lié à la supériorité par
interférence. Malgré la température élevé, les individus de grande taille
gagnant un accès quasi-exclusif à la ressource deviennent alors dominant dans
les popualtions. 

En plus de son effet direct sur l'individu, la température a donc également un
effet sur les interactions entre les individus et modifie les rapports de force
compétitifs entre individus de petite et de grande taille. En modifiant ces
rapports de force, la température affect de la dynamique des populations non
seulement via ses effets sur les individus, mais également via ses effets sur
les mécanismes de compétition. 

\subsection{Compétition par interférence dans les populations
naturelles}

Les différents résultats de ces travaux montrent l'importance de considérer la
compétition par interférence dans la description, la compréhension et la
prédiction de la dynamiques des populations structurées en taille. La taille
individuelle est un élément structurant des populations extrêmement répandu dans
les populations naturelles. Il est donc important de pouvoir reconnaître dans
ces populations les effets de la compétition par interférence. Les données
expérimentales présentées dans ces travaux rendent assez facile la détection de
la compétition par interférence et l'analyse de son rôle dans la dynamique des
populations. Malheureusement, ces données très complètes sont difficiles voir
impossible à recueillir dans la nature. Il faut donc pouvoir se baser sur des
critères plus simples permettant de proposer la compétition par interférence
comme mécanisme participant à la régulation des populations. 

A cette fin, notre étude théorique nous a permis de proposer des critères de
recherche pour identifier le rôle de la compétition par interférence. Parmi ces
critères, l'émergence d'individus géants dans les populations. Ces individus ont
alors une durée de vie très longue comparée aux individus plus petits, et
peuvent dominer les dynamiques de populations. Bien que plusieurs mécanismes
puissent engendrer des géants (cannibalisme, niche écologique spécifique,
\ldots), la présence des géants un premier indice qui, associé aux critères
suivant, pointe vers la compétition par interférence.

Si les données recueillis le permettent, l'observation de courbes de croissance
en deux temps avec une stagnation (proche de la maturation) suivi d'une reprise
de croissance avant de converger vers une taille asymptotique élevée est
également une indice d'une compétition par interférence relativement intense. Ce
ralentissement de croissance (goulet d'étranglement de la croissance) provoque
alors des distributions de taille très asymétriques, fortement biaisées en
faveur des juvéniles dans le cas ou la population est stable. Si la structure de
la population est cyclique avec de longue périodes où la distribution est
multi-modale, cela indique un niveau élevé de compétition par interférence. 

Enfin, dans le cas où les données ne permettent pas l'observation détaillée des
courbes de croissance ou de la dynamique temporelle de la structure des
populations, la mesure du temps de génération comparée à la durée des
oscillations dans des populations périodiques (les oscillations sont mesurées en
comptant l'abondance totale de la population) permet également de proposer la
compétition par interférence comme mécanisme régulateur si le ratio des deux
grandeurs est entre 1 et 2. Ce dernier indice n'apporte pas la même fiabilité
que les critères précédent, mais peut permettre de poser l'hypothèse de la
compétition par interférence et de diriger les efforts de récolte des données
nécessaires à l'identification de son rôle, telles que les courbes de
croissances individuelles ou les données détaillées de structure au cours du
temps. 

\section{Hypothèses et limites}

Les résultats rapportés dans cette thèse sont issue de travaux expérimentaux et
théoriques. Comme tous les travaux de ce type, mes travaux comptent leur lot de
limitations et d'hypothèses sous-jacentes. Nous allons tenter ici de les
expliciter afin de montrer comment en tenir compte et éventuellement les
contourner ou les lever dans de futurs développements. 

\subsection{Approches expérimentales}

Les différentes expériences menées au cours de cette thèse on consister à suivre
des collemboles dans des conditions contrôlées. Les conditions d'élevages et de
mesure des populations entraînent quelques limitations qu'il est nécessaire de
mentionner.

\subsubsection{Conditions d'élevage}

Nos collemboles sont élevés dans des boites en plastiques de cylindriques de
$5.1cm$ de diamètre remplies d'un substrat de $3cm$ permettant de conserver
l'environnement d'élevage avec un taux de saturation en eau proche de $100\%$.
En effet, les collemboles sont extrêmement sensibles à la dessiccation. Ceci
constitue une première difficulté dans l'élevage et le suivi de populations de
collemboles. Cela nous oblige à surveiller régulièrement l'état de sécheresse
des boites d'élevage. En cas de boite sèche, la survie des collemboles n'excède
pas quelques jours, voir quelques heures à température élevée. Il est donc
essentiel de maintenir le taux d'humidité à son maximum pour éviter une
mortalité accrue des individus. 

Une autre limite intrinsèque aux conditions d'élevage des collemboles est liée à
l'humidité nécessaire à la survie des collemboles. En effet, ces conditions
d'élevages et l'apport régulier de ressources est propice au développement de
moisissures sur la pastille de ressources et le fond de la boite. Durant leur
développement, ces champignons produisent de longs filaments dans lesquels les
collemboles se retrouvent piégés, et qui occupent parfois l'intégralité de la
pastille de ressources. Les collemboles sont alors incapables de se nourrir,
même s'ils sont toujours libre. Ces champignons provoquent donc une mortalité
accrue chez les collemboles, soit en occupant la ressource, soit en piégeant les
individus. L'invasion d'une boite par des champignons touche principalement les
individus les plus petits qui sont plus facilement piégés dans les filaments. La
mortalité accrue est donc plus grande chez les juvéniles. De plus, une fois
installé, il est extrêmement difficile de se débarrasser des champignons. La
contamination d'une boite nécessite donc de transférer la population dans une
nouvelle boite d'élevage seine. Ceci peut alors entraîner des perturbations des
dynamiques, et est à l'origine de certains des événements catastrophiques
décrits dans le Chapitre \ref{chap:sp}. Une solution possible pour éviter la
contamination par les champignons peut être d'utiliser des pastilles de
nourritures traitées aux fongicides, mais l'innocuité de ce traitement sur les
collemboles n'est pas démontré, et nous avons choisi de ne pas l'utiliser dans
nos expériences. A la place, nous avons prêté une attention particulière à
l'état de nos boites d'élevage qui ont été nettoyées régulièrement. Malgré tout,
il a été impossible de prévenir toute contamination, et les populations
atteintes ont été transférées dans de nouvelles boites d'élevages dès que la
contamination a été constatée. 

\subsubsection{Protocole expérimental}

Les suivis de populations mis en place au cours de cette thèse reposent sur un
apport hebdomadaire de nourriture sous la forme d'une pastille de levure
dissoute dans de l'agar-agar. Sous cette forme, la pastille de nourriture occupe
très peu d'espace sur le fond de la boite d'élevage. Comme nous l'avons montré
dans cette thèse, la compétition pour l'accès à ces ressources est très forte et
le petit espace occupé par la pastille favorise les interactions et la
compétition par interférence. Ceci nous a permis de décrire les effets de ce
type de compétition sur les populations structurées, mais il est possible qu'une
partie des résultats décrits dans cette thèse soient la conséquence d'un niveau
extrême de compétition par interférence, du au format de ressources choisi. Bien
que cela ne remette pas en cause les résultats décrits, il est nécessaire de
tenir compte de ces conditions particulières dans l'analyse des résultats
obtenus. 

\subsubsection{Méthode de mesure}

Les populations suivis ont été dénombrées et leur structure en taille mesurées
par la méthode semi-automatique décrite dans le Chapitre \ref{chap:method}.
Comme nous l'avons décrit, cette méthode est particulièrement fiable pour des
populations de moins de 1000 individus. Les populations suivis au
cours des différentes expériences ont atteint et parfois dépassé ces densités.
il est alors possible que les mesures de la structure des populations soient
sous estimées. Plus précisément, lorsque la densité d'individus est très
élevées, les individus se touchent ou se superposent sur les photos utilisées
pour les dénombrements et les mesures de tailles. Ainsi, au lieu de compté
plusieurs individus de différentes tailles, l'algorithme d'analyse d'image
compte une seule particule de grande taille. Il est donc possible à très grande
densité que le nombre de juvénile soit sous estimé et le nombre de grands
adultes surestimé. Cependant, les comptages réguliers des boites et les
mesures moyennées sur plusieurs photos permettent d'augmenter la confiance dans
les mesures obtenues. De plus, malgré les incertitudes de mesure, la
représentation sous forme de diagramme structure-temps permet une représentation
de la dynamique de la structure qui rend apparent les dynamiques à courts termes
et les motifs sur le long terme en lissant en partie le bruit issue des données. 

\subsection{Hypothèses du modèle}

L'analyse théorique du rôle de l'interférence dans la dynamique des populations
structurées repose elle aussi sur un certain nombre d'hypothèse dont il faut
tenir compte. 

\subsubsection{Budget énergétique dynamique et allocation de l'énergie}

Une première hypothèse du modèle choisi repose sur la règle du
budget énergétique dynamiques. Cette règle suppose notamment que l'absorption
d'énergie se fait proportionnellement à la longueur au carré de l'individu alors
que sa consommation se fait proportionnellement à la longueur au cube. Les
règles du budget énergétique dynamique ont été établies d'après des travaux sur
divers organismes par \textcites{kooijman2000a}. Chez les organismes étudiés,
l'ingestion de nourriture se fait par une bouche dont la taille est généralement
proportionnelle à la longueur de l'individu. Le taux d'ingestion de la
nourriture (et donc de l'énergie) dépend alors de la surface de la bouche, qui
est donc proportionnelle à la longueur au carré de l'individu. Le taux de
consommation dépend quand à lui du nombre de cellules présentes dans
l'organisme, ce qui dépend du volume de l'individu, essentiellement
proportionnel à sa longueur au cube. Bien que ces relations soient très
générales, elles n'ont pas été vérifiées précisément chez les collemboles à
cause des difficultés que posent la mesure de l'ingestion et de la consommation
de l'énergie chez ces organismes. Cependant, la grande généralité de cette
hypothèse fait que l'on peut la supposer également chez le collembole sans trop
de doute sur sa véracité. 

Associé à ces hypothèse sur l'ingestion de l'énergie et sa consommation vient
l'hypothèse sur l'allocation du budget énergétique. Dans les travaux présentés
dans le Chapitre \ref{chap:amnat}, nous faisons l'hypothèse d'une allocation
suivant la règle dite du $\kappa$. Cette hypothèse se justifie notamment par le
fait que les collemboles se reproduisent tout au long de leur vie, même après
avoir stoppé leur croissance. Cependant, afin de vérifier que les résultats du
modèles ne sont pas directement la conséquence de cette règle d'allocation de
l'énergie, nous avons testé une seconde règle communément utilisée dans les
modèles de populations structurées, la règle dite de ``production nette''
(``net production model''). Sous cette nouvelle hypothèse, l'énergie est d'abord
allouée à la maintenance, et le reste (s'il y en a) est divisé entre croissance
et reproduction. Une conséquence immédiate de cette règle est que l'arrêt de la
croissance s'accompagne nécessairement d'un arrêt de la reproduction. Ceci est
donc très différent de la règle précédemment utilisée. L'analyse du modèle dans
le cadre du modèle de production nette est présentée dans les Supplementary
Materials de l'Annexe \ref{An:AmNat}, Section \ref{subsec:SupMat4}. Les
résultats qualitatifs en terme de type de dynamiques obtenues pour les
différentes valeurs d'interférence sont sensiblement identiques à ceux obtenus
pour la règle du $\kappa$. Ces résultats sont donc robustes et ne dépendent pas
de la façon dont l'énergie est répartie après acquisition. 

\subsubsection{Reproduction continue}

Le modèle utilisé est une extension du modèle de \textcites{kooijman1984a}. Ce
modèle a été développé pour les Daphnies. Nous l'avons transposé aux collemboles
après avoir vérifier les points essentiels assurant la validité du modèle
(voir Section \ref{subsec:SupMat1} des Supplementary
Materials de l'Annexe \ref{An:AmNat}). 

Le modèle utilisé fait l'hypothèse d'une reproduction continue des individus.
Cependant, les collemboles se reproduisent en produisant des pontes
relativement importantes tous les $10$ à $20$ jours suivant les conditions de
température et de densité des populations. A l'échelle des simulations
réalisées (plusieurs milliers de jours), cela peut être considéré comme
quasiment continue, cependant, il est possible que l'éclosion massive
d'individus au même moment entraîne des résultats différents d'une population ou
la reproduction est strictement continue. Nous pensons toutefois que cette
hypothèse de reproduction continue, par opposition à la reproduction saisonnière
de certains modèles de poissons, notamment intégrant du cannibalisme
\autocites{claessen2000a,claessen2004a}, reste valide pour les collemboles
compte tenu des échelles de temps mis en jeu et du nombre d'individus présents
dans les populations. En effet, même si chaque individus se reproduit de par des
pontes, la présence de nombreux adultes entraîne des éclosions quasi-continues
dans les populations. 

\subsubsection{Représentation de la compétition par interférence}

Afin d'implémenter la compétition par interférence dans le modèle, et parce que
les données correspondantes n'étaient pas disponibles, nous avons du faire une
première approximation et nous affranchir de la description explicite des
ressources et de leur consommation. Nous avons remplacé la dynamique de la
ressource par une dépendance directe de la population à sa propre densité (via
la fonction $\eta$). Cette approximation a été réalisée en supposant la
ressource à un état quasi stationnaire. Or, telle qu'elle est apportée dans les
populations expérimentales, la ressource n'est pas à un état quasi-stationnaire.
Il pourrait donc être intéressant de mesurer la consommation de la ressource
chez les collemboles afin de relaxer cette hypothèse et réintégrer la ressource
dans le modèle. L'apport ponctuel de ressource pourrait alors avoir un effet
important sur les dynamiques produites par le modèle. 

De plus, la compétition par interférence a été implémentée via la fonction de
compétition $C$ qui défini la supériorité d'un individu sur un autre en fonction
des tailles relatives des contestants. Cette fonction a été choisie affine de
fonction arbitraire avec une pente définie par l'intensité $I$ de la
compétition. Ce choix a été motivé par une volonté de simplicité de la
représentation. Mais ceci rend le modèle relativement théorique, et pas
forcément facilement applicable directement à des organismes. De plus, cette
représentation ne prends pas en compte la possible dépense d'énergie par les
contestant dans les interactions qui les opposent. Ainsi l'intérêt d'accéder à
la ressource pourrait être limité si la dépense d'énergie nécessaire au gain de
cet accès est supérieure à l'énergie gagnée. Ceci pourrait être pris en compte
en attribuant une forme plus complexe à la fonction de compétition qui
intégrerait par exemple une dépendance à la taille absolue en plus de celle à la
taille relative à l'autre contestant. En supposant qu'un individu plus grand
dépense moins d'énergie qu'un petit dans un conteste, ou que des individus
proches en taille dépensent plus d'énergie que des individus très éloignés, on
pourrait également obtenir des dynamiques prédites différentes suivant les
niveaux d'interférence. Cependant, cela complexifie la
modélisation, et nécessiterait de nouvelles mesures pour choisir et calibrer les
fonctions de compétition. Or ces mesures sont potentiellement difficiles
réalisées (notamment sur le collembole). De plus, malgré les différences
quantitatives, les résultats qualitatifs issus des simulations apportent des
éléments d'explication des dynamiques observées dans les suivis de populations.
Nous pensons donc que le modèle décrit ici fourni une représentation simple et
minimaliste, mais pertinente des mécanismes de base de la compétition par
interférence et de ses conséquences sur la dynamique des populations
structurées. 

\section{Perspectives futures}

Comme tout travail scientifiques, les réponses aux questions soulevées dans
cette thèse ne s'arrêtent pas aux conclusions exposées dans ce chapitre. Au
contraire, comme le montrent les limites présentées précédemment, beaucoup de
travail reste à faire pour comprendre pleinement les impacts de la compétition
par interférence sur la dynamique des populations. De même, des zones d'ombre
persistent quant à l'interaction de la compétition par interférence avec les
autres mécanismes de densité dépendance, et avec les facteurs environnementaux
comme la température, ou les facteurs climatiques plus large, la structuration
de l'habitat, et tout autre facteur susceptible d'influencer les individus et
les populations. Afin de répondre au critiques présentées précédemment et
d'approfondir la connaissance du rôle de la compétition par interférence dans la
dynamique des populations, nous proposons ici quelques pistes à la fois
expérimentales et théoriques pouvant apporter de nouveaux éclairages sur ces
questions.

\subsection{Développement expérimental}

\subsubsection{Limiter les risques de contamination}

D'un point de vu expérimental, nous avons vu que plusieurs problèmes peuvent se
poser en lien avec le système développé au cours de cette étude. Un premier
impératif pour garantir la validité des résultats expérimentaux obtenus est de
trouver une méthode pour limiter au maximum les risques de contamination par les
champignons. Une méthode permettant d'assainir les boites d'élevage consiste à
étaler un mélange d'argile et de charbon actif, inoffensif pour les collemboles,
sur le fond de la boite d'élevage. Cela fourni un meilleur environnement de vie
pour les individus tout en diminuant le risque de contamination. Cette méthode
ayant été développé après le lancement de nos suivis de populations, nous ne
l'avons pas utilisé dans nos expériences, mais elle a été testée et vérifiée par
François \textcites{mallard2013b} dans ses travaux de thèse et peut être
appliquée aux futures expériences. 

\subsubsection{Tester les nouvelles questions}

Les travaux menés au cours de cette thèse on soulevés de nouvelles questions, en
particulier concernant le lien entre le niveau de compétition par interférence
et les structures observées dans les dynamiques. Nous avons évoqué le fait que
le niveau élevé de compétition par interférence dans nos populations pouvait
être lié à la distribution très localisée de la ressource dans les boites
d'élevage. Un moyen de réduire le niveau de compétition par interférence et
ainsi de tester l'effet sur la dynamique de la structure des populations par
comparaison avec les résultats précédent, pourrait être de disperser la
nourriture apporter. Il s'agirait alors de fournir la même quantité de
nourriture mais occupant une surface plus grande sur le fond de la boite.
L'hypothèse est alors qu'en occupant une proportion plus grande de la surface de
la boite, l'accès aux pastilles de nourriture est plus facile pour l'ensemble
des individus, ce qui permettrait de diminuer les interactions pour y accéder,
tout en maintenant un niveau similaire de compétition par exploitation en
gardant la même quantité totale de nourriture. A l'inverse, un traitement ou la
nourriture serait apportée en plus petite quantité mais plusieurs fois par
semaine afin de conserver la même quantité hebdomadaire permettrait de renforcer
la compétition par interférence. En effet, les pastilles seraient alors encore
plus petites que dans les conditions déjà testée, et les combats pour y accéder
plus intenses. 

De telles expériences à $21\degres$C permettraient de comparer les dynamiques
obtenues avec celles présentées dans le Chapitre \ref{chap:sp}. Ces
comparaisons permettraient également de tester les prédictions du modèle
présenté dans le Chapitre \ref{chap:amnat}, et ainsi, en faisant varier le
niveau de compétition par interférence, vérifier si l'on peut obtenir
expérimentalement les différentes dynamiques produites par les simulations.
Cette validation expérimentale fournirait alors un argument en faveur de la
représentation simple de l'interférence proposée dans le modèle, et permettrait
également de tester les critères proposés pour reconnaître le rôle de
l'interférence. 

Étendues aux différentes températures testées dans le Chapitre \ref{chap:fip},
ces expériences permettraient de vérifier l'hypothèse proposée d'une bascule
entre domination de la compétition par exploitation et par interférence entre
$21$ et $26\degres$C. En effet, nous supposons que les tailles plus grandes
observées à $26\degres$C sont dues à une forte intensité de l'interférence qui
donne un avantage à des individus de grande taille malgré le désavantage causé
par une température élevée. En faisant varier les niveaux d'interférence à
aux différentes températures, nous pourrions vérifier s'il est possible
d'observer un avantage aux petites taille jusqu'à $26\degres$C en diminuant
l'interférence, ou au contraire, favoriser l'émergence de géant dès
$21\degres$C, voir avant. Ceci permettrait de valider notre hypothèse et de
confirmer les interactions entre processus démographiques et effets de la
température dans la régulation des populations.

Une autre piste envisageable dans pour étendre notre étude de la compétition
concerne l'impact de la température dans le cas d'un environnement changeant. On
pourrait modifier les températures d'élevage des collemboles avec différentes
fréquences et amplitude, et ainsi observer comment les variations de température
impactent la dynamique des populations par comparaison avec les résultats
obtenus dans le Chapitre \ref{chap:fip} pour des températures fixes. On peut par
exemple imaginer que l'effet de la compétition par interférence favorisant les
grandes tailles à haute température associé à celui de la température elle-même
pour des températures inférieurs provoque une synergie si les variations
conduisent la population dans les deux régions. Ainsi, les adultes pourraient
avoir des tailles encore supérieures à celles observées.

\subsection{Développement théorique}

\subsubsection{Relâcher les hypothèses du modèle}

La première hypothèse du modèle théorique décrite précédemment concerne le
budget énergétique dynamique et le rapport $l^2 / l^3$ dans la gestion de
l'énergie. Cette hypothèse est longuement discutée par \textcites{kooijman2000a}
et nous avons déjà expliqué pourquoi nous la pensons applicable à notre système.
De même, nous avons montré que le choix de la règle d'allocation de l'énergie
n'avait pas d'impact direct sur les dynamiques prédites par notre modèle.
Cependant, d'autres hypothèses du modèle mériteraient d'être relâchées afin
d'en rendre les prédictions plus générales. 

Au premier rang de celles-ci se trouve l'hypothèse d'état quasi-stationnaire des
ressources. A cause des difficultés que posent la mesure des taux de
consommation et d'utilisation individuels des ressources chez les collemboles,
nous nous sommes restreint à cette hypothèse. Cependant, un retour à un modèle
intégrant explicitement la dynamique de la ressource améliorerait
significativement le réalisme de ce modèle. De plus, il deviendrait du
même coup plus facile à étendre à d'autres organismes, ou à d'autres types de
ressources. Nous pourrions par exemple tester l'impact d'un forçage sur les
ressources sur la dynamique des populations structurées sous différents niveaux
de compétition par interférence. 

La seconde hypothèse majeure concerne la forme de la fonction de compétition.
Comme nous l'avons expliqué précédemment, une forme différente de la fonction de
compétition pourrait mener à des résultats très différents sur les dynamiques
des populations prédites, notamment en intégrant un coup énergétique à
l'interférence. Il serait donc nécessaire de tester de nouvelles descriptions
pour l'interférence, et ainsi de vérifier la généralité des résultats proposés
dans cette thèse. 

\subsubsection{Développer le modèle}

Au delà du simple relâchement des hypothèses fortes du modèle, plusieurs pistes
sont envisageables pour prolonger l'étude théorique des effets de l'interférence
sur la dynamique des populations structurées. Le premier développement à
envisager consisterait à ajouter une dépendance des traits physiologiques à la
température. L'addition de la température dans le modèle, en la calibrant sur
les normes de réactions mesurées chez les individus, permettant alors de tester
les hypothèses formulées à l'issue de l'expérience présentée dans le Chapitre
\ref{chap:fip}. Notamment, en jouant sur le niveau d'interférence à différentes
température, on pourrait alors vérifier s'il est possible de prédire avec ce
modèle les normes de réactions moyennes observées dans nos populations
expérimentales, et notamment si une compétition par interférence
suffisamment intense conduit à une augmentation de la taille adulte maximum à
haute température. 

L'intégration de la température dans le modèle permettrait également de tester
et établir des prédictions de dynamiques dans le cas de températures
fluctuantes. Les effets de la températures étant en interaction avec ceux des
processus démographiques, l'impact de changements plus ou moins réguliers de
l'environnement est alors difficile à prévoir. Ces prédictions pourraient être
testées et validées expérimentalement comme décrit précédemment. 

Enfin, d'autres pistes pour le développement du modèle pourraient par exemple
consister à ajouter de l'évolution, et de se placer dans le contexte de la
dynamique adaptative pour tenter de prédire le devenir de populations soumises à
un changement extérieur selon leur capacité à s'adapter, par exemple au niveau
du compromis entre croissance et reproduction. Les différentes stratégies
possibles pouvant être testées sur notre système expérimental grâce à la banque
de clone disponible dont les phénotypes ont été décrits et étudiés
\autocites{tully2004a}.

Dans un contexte de changement globaux, notamment anthropogéniques, les travaux
présentés dans cette thèse apportent de nouveaux éléments à la compréhension
détaillée de la dynamiques des populations structurées, et ouvrent de nouvelles
perspectives dans l'analyse et la prédiction des réponses de ces populations aux
changements à venir.
