\partimage[width=\textwidth]{FigParts/writing3}
\part{Conclusion Générale}

\chapter{Dans une coque de noix}
\chaptermark{Dans une coque de noix}

\section{Principaux résultats de la thèse}

Au cours de cette thèse, nous avons exploré certains détails de la dynamique des
populations structurées, à la fois d'un point de vue expérimental que théorique.
Nous avons porté une attention toute particulière aux rôles que jouent les
interactions taille dépendantes entre individus dans l'établissement de ces
dynamiques, d'abord en observant des populations expérimentales de Collembole
\textit{Folsomia candida} sur le long terme pour en analyser les dynamiques en
fonction de leur structure en taille, puis en modélisant les interactions
tailles dépendantes entre individus dans un modèle physiologiquement structuré,
et enfin en manipulant la structure de la population et en observant les
comportement d'accès aux ressources. Nous nous sommes ensuite intéressé à
l'effet de l'environnement, via la température ambiante, sur ces interactions et
les répercussions sur la dynamique des populations.

Dans ce dernier chapitre, nous reviendrons sur les principaux enseignements que
l'on peut tirer de ces travaux par ses développements tant méthodologiques que
théoriques ou expérimentaux.

\subsection{Les développements méthodologiques}

J'ai choisi de revenir ici sur quelques développements méthodologiques réalisés
au cours de cette thèse car ils n'ont pas seulement occupé une grande partie de
mon travail, mais se sont également révélés indispensable au bon déroulement de
ces travaux. 

\subsubsection{Le phénotypage haut débit}

Le développement de la méthode d'analyse d'image pour le dénombrement et la
mesure de la structure des populations a été un élément clé qui a rendu possible
ce travail. Cette méthode de suivi des populations a été initialisées par
Thomas \textcites{tully2004a} au cours de sa thèse, puis reprise en développée
par François \textcites{mallard2013b} et moi même pendant plusieurs années avant
de faire l'objet d'un chapitre dans un ouvrage du CNRS
\autocites{le-galliard2012a} et d'une publication \autocites{mallard2013a}.

La méthode développé a notamment rendu possible la réplication des expériences
dans une mesure que l'on n'aura pas pu atteindre en dénombrant manuellement les
populations. C'est une méthode fiable qui permet à peu de frais d'accéder à la
structure en taille d'une population (et donc à la densité dans les différentes
classes de taille) en peu de temps et de façon très peu voir pas du tout
intrusive. 

Cette méthode a été développée et appliquée à des populations de collemboles,
mais est susceptible d'être facilement transposée à d'autres systèmes
expérimentaux comme cela a été décrit dans \textcites{mallard2013a}, ce qui lui
donne un grand intérêt pour l'écologie expérimentale et l'étude des traits
d'histoire de vie. 

\subsubsection{Diagrammes structure-temps}

Le travail présenté dans cette thèse n'aurait pas non plus été possible sans le
développement de la représentation graphique en diagrammes structure-temps et
des outils d'analyse graphique associés. En effet, cette technique simple de
représentation des données a permis de rendre cohérentes des données d'une
grande richesse dans les quelles il aurait été facile de se perdre. 

Nous pensons que l'application de cette méthode à des séries temporelles
structurées permettra de mettre en évidence des phénomènes qu'il était jusqu'à
présent difficile de décrire. Cette représentation rempli les critères
d'excellence des représentations graphiques en statistiques décrits par
\textcites{tufte1990a} en condensant une grande quantité d'information sur un
petit espace de représentation, sans déformer les données et en révélant
plusieurs niveaux de détails en une seul fois, des structures fines à la vue
d'ensemble \autocites{tufte2001a}. Avec le développement des méthodes
automatisées telles que celle présentée dans cette thèse \autocites[voir aussi
][]{le-galliard2012a} et la croissance toujours accélérée des bases de données
(``big data''), ce type de méthode devrait à l'avenir se démocratiser, autant en
écologie des populations que dans des domaines plus variés comme en
démographie humaine, en épidémiologie, dans les enquêtes d'opinion, d'usage,
\textit{etc}.

\subsection{Taille corporelle et populations structurées}

Au delà des apports méthodologiques, les travaux présentés dans cette thèse
répondent à des questions fondamentales en écologie des populations quant aux
rôle des interactions entre individus au sein d'une population dans la
régulation de sa structure et de sa dynamique, et à la place qu'occupe la taille
corporelle des individus dans la détermination de ces interactions.

\subsubsection{Taille corporelle et compétition par interférence}

Un premier résultat qui ressort de cet étude est l'importance de la différence
de taille corporelle dans la régulation de la dynamique de la structure en
taille de nos populations expérimentales. Nous avons pu montrer au cours des
différentes expériences menées et suivis de populations que la présence de
classes de tailles différentes en densité différentes avait un impact direct sur
la dynamique que l'on pouvait observer, tant à court terme qu'à long terme. La
présence d'individus de grande taille notamment s'est montré un critère
déterminant dans les dynamiques observées dans les différentes populations. 

En observant plusieurs populations dans les mêmes conditions, nous avons pu
montrer que les différentes structures en taille observées sont en nombre
limité, et peuvent être regroupées en quatre grande catégories. Ces catégories
de structures, que l'on a appelé structures types, se différencies par
l'abondance des juvéniles, la taille des adultes, et leur abondance. On peut
noter en particulier que l'une de ces catégories est particulièrement
remarquable par le fait qu'elle contient trois modes séparés dans la
distribution de la taille (les structures de types 4), contrairement aux trois
autres qui n'en contiennent que deux. Cette structure type n'a par ailleurs été
observée que chez un seul des deux clones étudiés, HA, et cela quelque soit
l'expérience menée. Mais le point commun entre les quatre types de structure est
la présence d'adultes à une taille corporelle stable nettement supérieure à la
taille à maturité, ce qui est contraire aux prédictions des modèles de
populations structurées régulées par compétition intra-spécifique par
exploitation. 

L'observation des dynamiques de court terme dans ces populations a permis de
démontrer que ces adultes de taille relativement élevée jouaient un rôle
déterminant dans les dynamiques de la structure des populations. Leur
disparition progressive ou brutale coïncide de façon quasi systématique avec un
événement de recrutement de juvéniles dans les classes adultes. La perturbation
de la structure des populations dans une seconde expérience a permis de
renforcer le lien de causalité entre la présence des adultes et la croissance ou
non des juvéniles. En effet, retirer l'ensemble des adultes d'une de nos
populations de collemboles provoque un événement massif de recrutement alors que
ne retirer qu'une partie des adultes, en particulier les plus petits, réduit
grandement le nombre de juvéniles qui parviennent à recruter. Retirer les
individus les plus grands permet donc de rendre la possibilité aux individus
plus petits de grandir et d'atteindre la maturité, puis de stabiliser leur
taille dans le mode des adultes les plus petits. 

Cette domination des adultes sur la dynamique de la structure de nos populations
s'explique par la domination qu'ils exercent sur l'accès à la ressource fournie.
Cette hypothèse, issue de l'observation des séries temporelle de la structure
des populations, a été confirmée par l'observation des comportements d'accès à
la ressource et la mesure du biais de taille dans cet accès. Par ces mesures, on
a pu montrer une sur-représentation systématique des individus les plus grands
au contact ou aux abords de la pastille de ressource. Cette pastille de
ressource étant très localisée comparé à l'environnement de vie des Collemboles,
il est alors facile pour la minorité d'individus les plus grands de monopoliser
la ressource et d'en restreindre l'accès aux plus petits. Ces derniers ne
pouvant se nourrir ne peuvent alors pas se développer et restent dans un stade
juvénile jusqu'à ce que le nombre d'adultes diminue suffisamment pour que la
ressource deviennent à nouveau accessible. 

Ce comportement des individus les plus grands repoussant les plus petits aux
abords de la ressource et ses conséquences sur la dynamique de la structure
montrent l'existence d'une compétition par interférence, parfois de forte
intensité, dans nos populations de collemboles. Cette compétition par
interférence contre balance l'avantage énergétique que possèdent les petits
individus et permets ainsi au plus grands de survivre et de dominer la
population. 

\subsubsection{Compétition par interférence et dynamique des populations
structurées}

Nous avons donc montré le rôle de la compétition par interférence dans la
régulation de nos populations structurées en taille, et notamment comment elle
permettait la survie d'individus de grande taille, et parfois de géants très
longévives qui, même en petit nombre, dominent le reste de la population. 

Afin d'étudier plus avant les conséquence de l'existence d'une compétition
intra-spécifique par interférence sur la dynamique d'une population structurée,
nous avons repris le modèle classique de dynamiques de populations
physiologiquement structurées développé pour les Daphnies
\autocites{kooijman1984a}. Nous avons adapté ce modèle à notre système en le
paramétrant pour les collemboles, ce qui a été possible car le cycle de vie des
collemboles présente des similarités avec celui des daphnies dans le sens où les
collemboles ont une croissance continue et sont amétaboles, les juvéniles et les
adultes ayant la même apparence et partageant la même niche, ne se différenciant
que par la taille. Nous avons ensuite étendu le modèle pour y intégrer une
représentation des interactions directes entre individus dépendantes des
tailles respectives des adversaires. Nous avons alors été en mesure de modéliser
différents niveaux de compétition par interférence pour en observer les
conséquences sur la dynamique, comparé à un modèle avec de la
compétition par exploitation seule. Ce modèle présente l'avantage de rester
simple tout en regroupant les aspects principaux de la compétition par
interférence.

Nous avons montré avec ce modèle que la présence de la compétition par
interférence intra-spécifique pouvait avoir différentes conséquences en fonction
de son intensité. En l'absence d'interférence, la théorie existante sur les
modèles PSP décrit déjà les dynamiques prédites, et notamment la présence de
cycles de génération à faible mortalité, et l'effet stabilisant d'une forte
mortalité. Un premier résultat de ce modèle est l'effet stabilisant de la
compétition par interférence lorsqu'elle est à niveau intermédiaire. Si son
intensité est suffisante, la compétition par interférence vient contrebalancer
l'avantage des juvéniles par la compétition par exploitation, et produit ainsi
un effet similaire à l'augmentation de la mortalité basale en stabilisant les
cycles de génération. On obtient alors une structure de population stable
similaire à celle que l'on aurait sans interférence avec une forte mortalité
basale. 

L'augmentation de la compétition par interférence a aussi pour effet de faire
augmenter la capacité des individus les plus grands de la population à accéder à
la ressource. Lorsque cette capacité est suffisante, les individus, qui jusque
là arrêtaient de grandir à une taille proche de la taille à maturité, peuvent
alors reprendre leur croissance, et atteignent des tailles importantes, proche
de leur maximum physiologique possible. Il est intéressant de noter que cela se
produit aussi bien dans une dynamique stable que cyclique. Une fois dépassé le
seuil d'interférence permettant la survie des géants, ceux-ci seront toujours
présents dans les populations et les domineront, à moins que le niveau
d'interférence soit réduit. 

Enfin, à un niveau très élevé, la compétition par interférence donne un tel
avantage aux adultes de grande taille que la dynamique est déstabilisée dans un
nouveau type de cycles. Ces cycles sont également des
cycles de génération, mais de beaucoup plus grande amplitude et plus longue
période. De plus contrairement aux cycles observés en l'absence d'interférence,
ce sont cette fois les adultes qui dominent la dynamique et génèrent les cycles. 

La présence des individus de grande taille et les nouveaux cycles décrits
présentent des similarités avec d'autres phénomènes déjà étudiés dans les
populations structurées, notamment le cannibalisme, ou la compétition
différentielle entre adultes et juvéniles, mais apporte une nouvelle explication
aux dynamiques observées, parfois plus parcimonieuse que celles précédemment
proposées. Ce modèle de dynamique de populations structurées, régulées par la
compétition par interférence, permet donc de compléter les résultats
préexistants sur la compétition par exploitation et le cannibalisme comme
mécanisme régulateur des popualtions structurées. 

\subsection{Les effets de la température}

\subsection{Observations en laboratoire -- détection dans les populations
naturelles}

\section{Hypothèses et limites}

\subsection{Approches expérimentales}

\subsubsection{Conditions d'élevage}

\subsubsection{Méthode de mesure}

\subsection{Hypothèses du modèle}

\section{Perspectives futures}