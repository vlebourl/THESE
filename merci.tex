\chapter*{Merci}

Déjà presque que quatre années de travail, comment remercier correctement tous
les gens qui m’ont aidé à en voir le bout ? Je vais essayer de n’oublier
personne, et je prie toutes les personnes qui n’apparaîtraient pas ici et s’en
sentiraient lésées de m’excuser d’avance, tellement m’ont aidé qu’il est
difficile de ne pas en oublier.

Je voudrais tout d’abord remercier les membres de mon jury de thèse, Andre De
Roos et Ophélie Ronce, pour avoir eu la patience et le courage d’arriver au bout
de ce manuscrit, et pour être venu à Paris m’aider à en tirer le meilleur. Je
remercie également Bruno Ernande et Amaury Lambert pour avoir bien voulu prendre
part à ce jury et m’évaluer à l’issue de ce long travail. Je remercie enfin
Jean-Christophe Poggiale qui en plus de venir m’écouter et m’évaluer à Paris, a
participé par ses conseils avisés lors de mes comités de thèse à faire de ce
travail ce qu’il est aujourd’hui. Je voudrais remercier également les autres
membres de mon comité de thèse, Tom Van Dooren, et Eric Edeline et Régis
Ferrière, pour leurs conseils éclairés et les longues discussions que nous avons
eu à plusieurs reprises, et pas seulement lors des dits comités.

J’aimerais remercier également Michel Volovitch qui m’a permis d’obtenir mon
monitorat et ainsi d’avoir une expérience extrêmement enrichissante
d’enseignement. Je voudrais à ce titre remercier Jean-François Le Galliard pour
m’avoir inclus dans ses TP d’écologie à Foljuif, ainsi que toute l’équipe de
Folljuif qui a fait de ces quelques jours une grande réussite pour les étudiants
autant que pour moi.

Je souhaiterais remercier Jean-François et Eric également pour l’aide qu’ils
m’ont apporté dans la mise en forme de ces travaux en participant à la relecture
de plusieurs manuscrits.

Je remercie bien sûr du fond du cœur mes deux directeurs, David Clæssen et
Thomas Tully. David, sans qui ma compréhension des différentes théories
écologiques serait encore très succincte. Grâce à toi, j’ai pu me plonger corps
et âme dans les méandres des populations structurées et tirer le maximum de ces
travaux. Tu m’as appris à conceptualiser, synthétiser, rendre mes idées claires
et savoir les partager. Je te remercie également de m’avoir fait participer à
tes enseignements, être confronté à des étudiants à peine plus jeunes que moi,
aux questions pointues que je n’avais pas anticipé m’a aussi beaucoup appris.
Enfin, tu as eu le talent de réunir une fine équipe de foot. Même si je n’y suis
pas resté longtemps, ces petits moments de détentes étaient des plus
appréciables. Thomas, tu m’as montré le monde de l’expérimentation et des
résultats empirique. Je n’aurais jamais imaginé qu’on puisse tirer tant de
résultats de si peu de données, et cela d’une expérience des plus simples. Tu
m’as appris à pousser l’ambition toujours un peu plus loin, à ne pas m’arrêter à
ce que j’avais devant les yeux. Tu as su alimenter et stimuler mon imagination.
Grâce à tes idées parfois farfelues, on pu avancer beaucoup plus loin que je
n’aurais su le faire seul. Enfin tu as toujours apprécié mon travail et su me
remotiver dans les moments de creux, parfois autour d’une petite bière chez
Youssef ou à la Montagne. Merci donc à vous deux pour m’avoir guidé, forgé et
élevé jusqu’ici. Sans vous cette thèse n’aurait jamais abouti.

Je voudrais remercier François pour son année d’avance sur moi, qui m’a permis
d’anticiper sur mon futur, pour m’avoir aidé à dompter les collemboles et tout
ce qui les concerne, pour m’avoir inclus dans une partie de ses travaux et pour
m’avoir fait découvrir le requin et sa table magique.

J’ai découvert pendant cette thèse la vie dans un laboratoire avec ses déboires
mais aussi ses bons moments. Je remercie tous les membres de l’équipe EEM de
m’avoir accueilli, de m’avoir fait partager leur savoir et de m’avoir écouté, et
particulièrement Jean-Baptiste pour les grands débats sur l’évolution. Je
remercie aussi tous les membres des laboratoires EcoEvo et BioEmco. Je remercie
plus particulièrement les thésards et les postdoc des deux labos, Melissa, Lisa,
Anaïs, Ewen, Sylvain, Thomas, Aleksandar, Benoit, Charlène, Grigoris, Loïc,
Aurore, Solène, et tous ceux que j’oublie. Un merci particulier à Romain et
Monique qui nous ont bien aidé à faire face à nos bestioles !

Je remercie également tous les stagiaires qui ont circulé dans notre antre à
collemboles, et qui nous ont bien aidé pour faire avancer nos travaux et venir
partager un petit verre après, Alexandre, Lorenza, Christie, Ariane, François,
Hugo et Claire. J’espère que vous avez apprécié autant que moi le travail en
notre compagnie.

Je remercie bien sûr mes collègues de bureau au sens large. Boris, toi qui
partage mon quotidien au boulot depuis bien trop longtemps, j’espère que ma
présence n’aura pas été trop dure à supporter. Merci à Ben d’avoir apporté un
peu de classe dans ce bureau. Merci Jonathan d’être venu rapporter un bout de
Bretagne. Merci à Stéphane, tu n’étais pas dans mon bureau mais c’était presque
pareil. Merci à Laurent et Sylvain pour l’instant sport de la semaine. Merci à
la MudTeam. Merci à Denis et toute l’équipe du CERES de m’avoir hébergé et
accueilli entre leur murs. Merci à l’ANR EVORANGE et au programme
interdisciplinaire sur la longévité pour avoir financé mes recherches.

Merci à Marc, Pauline, Diane, Clément, Jean et l’équipe de la Montagne.

Merci à tous mes amis. Un grand merci à mes colocs de la team BA, Clémence, si
tu n’avais pas pris soin de moi à la maison, je ne serais peut-être pas arrivé
au bout, Thomas, voir ton sourire tous les matins m’a donné l’énergie dont
j’avais besoin pour arriver là. Un merci particulier à Tino pour m’avoir fait
voir les étoiles quand j’avais les yeux rivés sur le sol, et à Louise de ne pas
m’avoir lâché après être rentré de Londres. Merci aussi à Florane, Lakshite,
Juliette, Malou, Coline. Merci à mes amis swingueurs, Eve, Adam, Garance,
Michel, Anteg, Céline, Amandine.

Merci à toute ma famille, et en particulier à mes parents, à Christophe et à
Marie d’avoir été là depuis le tout début. Sans vous rien n’aurait été possible.

Merci Iannick, tu as su tenir le rythme aussi bien sinon mieux que moi. Grâce à
toi, un page se tourne pour mieux commencer le prochain chapitre.


