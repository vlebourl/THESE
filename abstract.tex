\begin{abstract}
De plus en plus reconnue comme jouant un rôle majeur dans la régulation des
populations, la compétition par interférence et ses effets sur la dynamique des
populations suscitent un intérêt croissant. De plus, la température est connue
comme facteur abiotique ayant un fort impact sur la physiologie et les
comportements individuels ainsi que sur les dynamiques des populations.

Dans un contexte de réchauffement planétaire, comprendre comment les
interactions directes entre les individus se répercutent sur la dynamique
globale d’une population et comment cela interagit avec l’effet de la
température constitue un enjeu important pour la biologie des populations.

Les interactions interindividuelles étant fortement liées à la taille corporelle
des individus, la structure en taille de plusieurs populations de deux lignées
clonales du collembole Folsomia candida a été finement suivie pendant deux à
quatre ans à six températures de 6$\degres$C à 26$\degres$C. Une première analyse des séries
temporelles de la structure des populations à 21$\degres$C a révélé une dépendance forte
de la structure des populations et de sa dynamique aux conditions d’accès de
chaque individu à la ressource, liées notamment à la présence d’individus
adultes de grande taille. Nous avons ensuite modifié artificiellement la
structure de plusieurs populations en isolant certaines classes de taille dans
des populations séparées, et observé le retour à l’équilibre de la structure en
taille. Parallèlement, nous avons observé en temps réel le comportement d’accès
à la ressource. Ces études ont permis de montrer le rôle déterminant joué par la
présence d’adultes de grande taille dans la régulation de la population en
interférant avec les individus plus petit et en monopolisant la ressource.

La compétition par interférence a donc été implémentée dans un modèle théorique
de dynamique des populations structurées pour étudier l’impact du niveau
d’interférence sur la dynamique d’une population. Nous avons montré que
l’intensité de l’interférence peut avoir des effets variés sur la dynamique
d’une population structurée, tels que stabiliser des cycles de générations,
permettre la survie d’individus de très grande taille, et causer une
déstabilisation vers des cycles de très grande période et amplitude.

Nous avons ensuite comparé des normes de réactions à la température sur des
individus isolées et dans les populations suivies afin de comprendre comment la
compétition interagit avec la température dans la régulation de la dynamique de
la population. Ceci a également permis de montrer la nécessité d’englober
plusieurs niveaux de complexité pour comprendre comment des changements
environnementaux peuvent se répercuter sur la dynamique des populations. Cette
étude a enfin servi de base à une intégration de la température dans le modèle
précédemment développé afin d’étudier d’un point de vue théorique les
interactions entre température et compétition.

\vspace{1pt}

Mots clés: Collembole, dynamique des populations, populations structurées,
compétition, exploitation, interférence, modèles physiologiquement structurés,
température, norme de réaction

\end{abstract}